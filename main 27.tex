\documentclass[a4paper,12pt]{extreport}
\usepackage[utf8]{inputenc}
\usepackage[russian]{babel}
\usepackage{ragged2e}
\usepackage[mag=1000,a4paper,left=1cm,right=1cm,top=1cm,bottom=1cm,noheadfoot]{geometry}
\usepackage{mathtext}
\usepackage{amsmath,amssymb,amsthm,amscd,amsfonts,graphicx,epsfig,textcomp,wrapfig}
\usepackage[dvips]{graphicx}
\graphicspath{{noiseimages/}}

\newcommand{\tab}{\hspace{10mm}}
\newcommand{\name}{
			\normalsize
			{
				\ \ \quad
                \mbox{} \hfil {\flushleft{Взлет сборы}} \hfill {09.11.2023}
			%\quad
			}
			\vspace{5pt}\hrule
		}

\def\head#1#2{
	\begin{center}{
			\LARGE
			\bf #2
			
			{\normalsize \bf #1}
			\vspace{-10pt}
	}\end{center}
}

\newcommand{\del}{\mathop{\raisebox{-2pt}{\vdots}}}
\newcommand{\q}{}

\newcounter{tasknum}
\setcounter{tasknum}{0}
\def\thetasknum{{\textbf{\arabic{tasknum}}}}
\newcommand{\task}{\refstepcounter{tasknum}\vspace{2pt}\noindent \textbf{} \thetasknum\textbf{.}~}
\newcommand{\coff}{\refstepcounter{tasknum}\vspace{2pt}\noindent \textbf{Задача} \thetasknum *\textbf{.}~}


\newcommand{\eq}[1]{\underset{#1}{\equiv}}
\newcommand{\dv}{\ensuremath{\mathop{\raisebox{-2pt}{\vdots}}}}
\newcommand{\ndv}{\not \dv}

\newcounter{exnum}
\setcounter{exnum}{0}
\def\theexnum{{\emph{\arabic{exnum}}}}
\newcommand{\ex}{\refstepcounter{exnum}\vspace{2pt}\noindent \emph{Упражнение} \theexnum\emph{.}~}



\theoremstyle{plain}
\newtheorem{theorem}{Теорема}
\newtheorem{lemma}{Лемма}
\newtheorem{proposition}{Предложение}
\newtheorem{corollary}{Следствие}
\theoremstyle{definition}
\newtheorem{definition}{Определение}
\theoremstyle{remark}
\newtheorem{remark}{Замечание}
\newtheorem{example}{Пример}

\textheight=250mm %
\textwidth=180mm %
\oddsidemargin=-10.4mm%
\evensidemargin=-10.4mm %
\topmargin=-24.4mm


\begin{document}
	\pagestyle{empty}
    \name{}
	\head{9 класс}{Кон. По. Ин.}
	\bigskip
	
\task Докажите, что для любого $n$ найдется $n$ подряд идущих составных чисел.\\

\task Докажите, что для любого $k$ найдется квадратная клетчатая доска, на которой можно расставить $k$ не бьющих друг друга ферзей.\\

\task От прямоугольника с неравными сторонами отрезают квадрат со стороной, равной меньшей стороне прямоугольника. Если оставшаяся часть не квадрат, процесс повторяют. Докажите, что для любого $n$ найдется прямоугольник, для которого процесс закончится ровно после $n$-го отрезания, причем все отрезанные квадраты будут разного размера.\\

\task На столе стоят $2^{n}$ стаканов с водой. Разрешается взять один из стаканов и перелить из него часть воды в стакан, где воды меньше так, чтобы воды стало поровну. Докажите, что такими операциями можно добиться, чтобы во всех стаканах стало поровну воды.\\

\task На клетчатой доске стоит $n$ ферзей. Докажите, что их можно раскрасить в $5$ цветов так, чтобы ферзи одного цвета друг друга не били.\\

\task Докажите, что для любого $n$ найдется убывающая арифметическая прогрессия из $n$ членов вида $\frac{1}{k}$.\\

\task На доске выписаны числа $1, 2^{1}, 2^{2}, 2^{3}, ..., 2^{n}$. Разрешается стереть любые два числа и вместо них выписать их разность – неотрицательное число. После нескольких таких операций на доске будет только одно число. Чему оно может быть равно?\\

\task В шахматном турнире каждый с каждым сыграли по разу. Докажите, что можно так занумеровать участников, чтобы каждый не проиграл участнику со следующим номером.\\

\task Имеется много карточек, на каждой из которых записано натуральное число от $1$ до $n$. Известно, что сумма чисел на всех карточках равна  $n! \cdot k$,  где $k$ – целое число. Доказать, что карточки можно разложить на $k$ групп так, чтобы в каждой группе сумма чисел, написанных на карточках, равнялась $n!$\\

\end{document}
