\documentclass[a4paper,12pt]{extreport}
\usepackage[utf8]{inputenc}
\usepackage[russian]{babel}
\usepackage{ragged2e}
\usepackage[mag=1000,a4paper,left=1cm,right=1cm,top=1cm,bottom=1cm,noheadfoot]{geometry}
\usepackage{mathtext}
\usepackage{amsmath,amssymb,amsthm,amscd,amsfonts,graphicx,epsfig,textcomp,wrapfig}
\usepackage[dvips]{graphicx}
\graphicspath{{noiseimages/}}

\newcommand{\tab}{\hspace{10mm}}
\newcommand{\name}{
			\normalsize
			{
				\ \ \quad
                \mbox{} \hfil {\flushleft{кружок 5 школа}} \hfill {22.04.2023}
			%\quad
			}
			\vspace{5pt}\hrule
		}

\def\head#1#2{
	\begin{center}{
			\LARGE
			\bf #2
			
			{\normalsize \bf #1}
			\vspace{-10pt}
	}\end{center}
}

\newcommand{\del}{\mathop{\raisebox{-2pt}{\vdots}}}
\newcommand{\q}{}

\newcounter{tasknum}
\setcounter{tasknum}{0}
\def\thetasknum{{\textbf{\arabic{tasknum}}}}
\newcommand{\task}{\refstepcounter{tasknum}\vspace{2pt}\noindent \textbf{} \thetasknum\textbf{.}~}
\newcommand{\coff}{\refstepcounter{tasknum}\vspace{2pt}\noindent \textbf{Задача} \thetasknum *\textbf{.}~}


\newcommand{\eq}[1]{\underset{#1}{\equiv}}
\newcommand{\dv}{\ensuremath{\mathop{\raisebox{-2pt}{\vdots}}}}
\newcommand{\ndv}{\not \dv}

\newcounter{exnum}
\setcounter{exnum}{0}
\def\theexnum{{\emph{\arabic{exnum}}}}
\newcommand{\ex}{\refstepcounter{exnum}\vspace{2pt}\noindent \emph{Упражнение} \theexnum\emph{.}~}



\theoremstyle{plain}
\newtheorem{theorem}{Теорема}
\newtheorem{lemma}{Лемма}
\newtheorem{proposition}{Предложение}
\newtheorem{corollary}{Следствие}
\theoremstyle{definition}
\newtheorem{definition}{Определение}
\theoremstyle{remark}
\newtheorem{remark}{Замечание}
\newtheorem{example}{Пример}
\newtheorem*{idea}{Идея}

\textheight=250mm %
\textwidth=180mm %
\oddsidemargin=-10.4mm%
\evensidemargin=-10.4mm %
\topmargin=-24.4mm


\begin{document}
	\pagestyle{empty}
    \name{}
	\head{5 класс}{Принцип крайнего}
	\bigskip
	
\task По кругу выписано несколько натуральных чисел, каждое из которых не превосходит хотя бы одного из своих соседей. Докажите, что найдутся два равных7\\

\task По кругу выписано несколько натуральных чисел, каждое из которых равно полусумме своих соседей. Докажите, что все эти числа равны.\\

\task На шахматной доске стоит несколько ладей. Обязательно ли найдется ладья, бьющая не более двух других?\\

\task На плоскост отмечено $10$ различных точек. Докажите, что их можно разбить на пары и соединить внутри пар отрезками так, чтобы полученные отрезки не пересекались.\\

\task Космноват решил отправиться в экспедицию по планетам галактики $N$. Каждый раз с планеты, на которой он находится в данный момент, он летит на самую далекую от нее. Докажите, что если через $2$ перелета он не вернулся домой, то уже никогда не вернется. (Все расстояния между планетами различны).\\

\task В стране $N$ есть несколько городов. В каждом из них назначили мэра, который должен следить за порядком в своем городе. Но, как известно, у соседа трава зеленее, так что каждый мэр вместо работы наблюдает за ситуацией в ближайшем от него городе. Докажите, что за каким-то городом никто не наблюдает. (Все расстояния между городами различны).\\

\task На плоскости дано несколько многоугольников (не обязательно выпуклых), каждые два из которых имеют общую точку. Докажите, что можно провести прямую, которая будет иметь общую точку с каждым из многоугольников.\\

\task Однажды на консиллиуме любителей узлов побывало $30$ ученых, причём если ученый уходил с консилиума, то больше на него не возвращался. Оказа- лось, что среди любых трёх ученых какие-то двое встретились. Докажите, что организатор мог выступить с докдадом не более двух раз, чтобы его услышали все ученые.\\

\task В клетках таблицы размером $10 \times 20$ расставлено $200$ различных чисел. В каждой строчке отмечены два наибольших числа красным цветом, а в каждом столбце отмечены два наибольших числа синим цветом. Доказать, что не менее трёх чисел отмечены в таблице как красным, так и синим цветом.\\

\end{document}
