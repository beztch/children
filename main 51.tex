\documentclass[a4paper,12pt]{extreport}
\usepackage[utf8]{inputenc}
\usepackage[russian]{babel}
\usepackage{ragged2e}
\usepackage[mag=1000,a4paper,left=1cm,right=1cm,top=1cm,bottom=1cm,noheadfoot]{geometry}
\usepackage{mathtext}
\usepackage{amsmath,amssymb,amsthm,amscd,amsfonts,graphicx,epsfig,textcomp,wrapfig}
\usepackage[dvips]{graphicx}
\graphicspath{{noiseimages/}}

\newcommand{\tab}{\hspace{10mm}}
\newcommand{\name}{
			\normalsize
			{
				\ \ \quad
                \mbox{} \hfil {\flushleft{5 лицей}} \hfill {06.02.2024}
			%\quad
			}
			\vspace{5pt}\hrule
		}

\def\head#1#2{
	\begin{center}{
			\LARGE
			\bf #2
			
			{\normalsize \bf #1}
			\vspace{-10pt}
	}\end{center}
}

\newcommand{\del}{\mathop{\raisebox{-2pt}{\vdots}}}
\newcommand{\q}{}

\newcounter{tasknum}
\setcounter{tasknum}{0}
\def\thetasknum{{\textbf{\arabic{tasknum}}}}
\newcommand{\task}{\refstepcounter{tasknum}\vspace{2pt}\noindent \textbf{} \thetasknum\textbf{.}~}
\newcommand{\coff}{\refstepcounter{tasknum}\vspace{2pt}\noindent \textbf{Задача} \thetasknum *\textbf{.}~}


\newcommand{\eq}[1]{\underset{#1}{\equiv}}
\newcommand{\dv}{\ensuremath{\mathop{\raisebox{-2pt}{\vdots}}}}
\newcommand{\ndv}{\not \dv}

\newcounter{exnum}
\setcounter{exnum}{0}
\def\theexnum{{\emph{\arabic{exnum}}}}
\newcommand{\ex}{\refstepcounter{exnum}\vspace{2pt}\noindent \emph{Упражнение} \theexnum\emph{.}~}



\theoremstyle{plain}
\newtheorem{theorem}{Теорема}
\newtheorem{lemma}{Лемма}
\newtheorem{proposition}{Предложение}
\newtheorem{corollary}{Следствие}
\theoremstyle{definition}
\newtheorem{definition}{Определение}
\theoremstyle{remark}
\newtheorem{remark}{Замечание}
\newtheorem{example}{Пример}

\textheight=280mm %
\textwidth=180mm %
\oddsidemargin=-10.4mm%
\evensidemargin=-10.4mm %
\topmargin=-24.4mm


\begin{document}
	\pagestyle{empty}
    \name{}
	\head{6 класс}{Числа и суммы}
	\bigskip

\task Найдите наибольшее 10-значное число, состоящее из разных цифр, которое делится на 4.\\

\task Натуральные числа от 1 до 10 разбили на две группы так, что произведение чисел в первой группе делится на произведение чисел во второй группе. Какое наименьшее значение может принимать частное от деления первого произведения на второе?\\
	
\task Во всех подъездах дома одинаковое число этажей, а на каждом этаже одинаковое число квартир. При этом число этажей в доме больше числа квартир на этаже, число квартир на этаже больше числа подъездов, а число подъездов больше одного. Сколько этажей в доме, если всего в нём 105 квартир?\\

\task Шехерезада стала учительницей математики и задала школьникам на дом 1001 задачу. За каждую решённую задачу она начисляла 2 балла, за каждую неправильно решённую задачу штрафовала на один балл, а за каждую задачу, которую школьник не решал, штрафовала на пятьдесят баллов. Шахрияр правильно решил меньше 900 задач и набрал 1514 баллов. Сколько задач правильно решил Шахрияр?\\

\task Джеку Воробью нужно было разложить 150 пиастров по 10 кошелькам. После того как он положил некоторое количество пиастров в первый кошелёк, в каждый следующий он клал больше, чем в предыдущий. В результате оказалось, что количество пиастров в первом кошельке не меньше, чем половина количества пиастров в последнем. Сколько пиастров находится в шестом кошельке?\\

\task Петров забронировал квартиру в доме-новостройке, в котором пять одинако- вых подъездов. Изначально подъезды нумеровались слева направо, и квартира Петрова имела номер 636. Потом застройщик поменял нумерацию на противоположную (справа налево, т.е. первый подъезд стал пятым, второй — четвёртым, ..., пятый — первым) и соответствующим образом перенумеровал квартиры. (Порядок нумерации квартир внутри подъезда не изменялся.) Тогда квартира Петрова стала иметь номер 242. Сколько квартир в доме?\\

\task Тридцать три богатыря нанялись охранять Лукоморье за 240 монет. Хитрый дядька Черномор вправе разделить богатырей на отряды произвольной численности (в том числе — может записать всех в один отряд), а затем распределить всё жалованье между отрядами. Каждый отряд делит свои монеты поровну, а если нацело не разделится, то остаток забирает Черномор. Какое наибольшее количество монет может достаться Черномору, если жалованье между отрядами Черномор распределяет как ему угодно?\\

\task На едином экзамене 333 ученика допустили в общей сложности 1000 ошибок. Возможно ли при этом, что учеников, сделавших более чем по 5 ошибок, оказалось больше, чем учеников, сделавших менее чем по 4 ошибки?\\

\task Кольцевая дорога поделена столбами на километровые участки, и известно, что количество столбов чётно. Один из столбов покрашен в жёлтый цвет, другой – в синий, а остальные – в белый. Назовем расстоянием между столбами длину кратчайшей из двух соединяющих их дуг. Найдите расстояние от синего столба до жёлтого, если сумма растояний от синего столба до белых равна 2008 км.\\

\end{document}
