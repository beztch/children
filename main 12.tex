\documentclass[a4paper,12pt]{extreport}
\usepackage[utf8]{inputenc}
\usepackage[russian]{babel}
\usepackage{ragged2e}
\usepackage[mag=1000,a4paper,left=1cm,right=1cm,top=1cm,bottom=1cm,noheadfoot]{geometry}
\usepackage{mathtext}
\usepackage{amsmath,amssymb,amsthm,amscd,amsfonts,graphicx,epsfig,textcomp,wrapfig}
\usepackage[dvips]{graphicx}
\graphicspath{{noiseimages/}}

\newcommand{\tab}{\hspace{10mm}}
\newcommand{\name}{
			\normalsize
			{
				\ \ \quad
                \mbox{} \hfil {\flushleft{взлет МО}} \hfill {11.09.2023}
			%\quad
			}
			\vspace{5pt}\hrule
		}

\def\head#1#2{
	\begin{center}{
			\LARGE
			\bf #2
			
			{\normalsize \bf #1}
			\vspace{-10pt}
	}\end{center}
}

\newcommand{\del}{\mathop{\raisebox{-2pt}{\vdots}}}
\newcommand{\q}{}

\newcounter{tasknum}
\setcounter{tasknum}{0}
\def\thetasknum{{\textbf{\arabic{tasknum}}}}
\newcommand{\task}{\refstepcounter{tasknum}\vspace{2pt}\noindent \textbf{} \thetasknum\textbf{.}~}
\newcommand{\coff}{\refstepcounter{tasknum}\vspace{2pt}\noindent \textbf{Задача} \thetasknum *\textbf{.}~}


\newcommand{\eq}[1]{\underset{#1}{\equiv}}
\newcommand{\dv}{\ensuremath{\mathop{\raisebox{-2pt}{\vdots}}}}
\newcommand{\ndv}{\not \dv}

\newcounter{exnum}
\setcounter{exnum}{0}
\def\theexnum{{\emph{\arabic{exnum}}}}
\newcommand{\ex}{\refstepcounter{exnum}\vspace{2pt}\noindent \emph{Упражнение} \theexnum\emph{.}~}



\theoremstyle{plain}
\newtheorem{theorem}{Теорема}
\newtheorem{lemma}{Лемма}
\newtheorem{proposition}{Предложение}
\newtheorem{corollary}{Следствие}
\theoremstyle{definition}
\newtheorem{definition}{Определение}
\theoremstyle{remark}
\newtheorem{remark}{Замечание}
\newtheorem{example}{Пример}

\textheight=250mm %
\textwidth=180mm %
\oddsidemargin=-10.4mm%
\evensidemargin=-10.4mm %
\topmargin=-24.4mm


\begin{document}
	\pagestyle{empty}
    \name{}
	\head{9 класс}{Двойной подсчет}
	\bigskip
	
\task Можно ли занумеровать рёбра куба натуральными числами от $1$ до $12$ так, чтобы для каждой вершины куба сумма номеров рёбер, которые в ней сходятся, была одинаковой?\\

\task Тридцать школьников — семиклассников и восьмиклассников — обменялись рукопожатиями. При этом оказалось, что каждый семиклассник пожал руку восьми восьмиклассникам, а каждый восьмиклассник пожал руку семи семиклассникам. Сколько было семиклассников и сколько восьмиклассников?\\

\task Дано $3003$ числа. Известно, что сумма любых четырёх чисел положительна. Верно ли, что сумма всех чисел положительна?\\

\task По кругу расставлены цифры $1, 2, 3, . . . , 9$ в произвольном порядке. Каждые три цифры, стоящие подряд по часовой стрелке, образуют трёхзначное число. Найдите сумму всех девяти таких чисел.\\

\task Обозначим через $d_{k}$ количество таких домов в Москве, в которых живет не меньше $k$ жителей, и через $c_{m}$ - количество жителей в $m$-ом по величине населения доме. Докажите равенство $c_{1}+c_{2}+c_{3}+... = d_{1}+d_{2}+d_{3}+...$ .\\

\task Во взводе $10$ человек. В каждый из $100$ дней какие-то четверо назначались дежурными. Докажите, что какие-то двое были вместе на дежурстве не менее $14$ раз.\\

\task Дан набор, состоящий из таких $2023$ чисел, что если каждое число в наборе заменить на сумму остальных, то получится тот же набор. Докажите, что произведение чисел в наборе равно $0$.\\

\end{document}
