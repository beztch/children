\documentclass[a4paper,12pt]{extreport}
\usepackage[utf8]{inputenc}
\usepackage[russian]{babel}
\usepackage{ragged2e}
\usepackage[mag=1000,a4paper,left=1cm,right=1cm,top=1cm,bottom=1cm,noheadfoot]{geometry}
\usepackage{mathtext}
\usepackage{amsmath,amssymb,amsthm,amscd,amsfonts,graphicx,epsfig,textcomp,wrapfig}
\usepackage[dvips]{graphicx}
\graphicspath{{noiseimages/}}

\newcommand{\tab}{\hspace{10mm}}
\newcommand{\name}{
			\normalsize
			{
				\ \ \quad
                \mbox{} \hfil {\flushleft{взлет}} \hfill {01.10.2024}
			%\quad
			}
			\vspace{5pt}\hrule
		}

\def\head#1#2{
	\begin{center}{
			\LARGE
			\bf #2
			
			{\normalsize \bf #1}
			\vspace{-10pt}
	}\end{center}
}

\newcommand{\del}{\mathop{\raisebox{-2pt}{\vdots}}}
\newcommand{\q}{}

\newcounter{tasknum}
\setcounter{tasknum}{0}
\def\thetasknum{{\textbf{\arabic{tasknum}}}}
\newcommand{\task}{\refstepcounter{tasknum}\vspace{2pt}\noindent \textbf{} \thetasknum\textbf{.}~}
\newcommand{\coff}{\refstepcounter{tasknum}\vspace{2pt}\noindent \textbf{Задача} \thetasknum *\textbf{.}~}


\newcommand{\eq}[1]{\underset{#1}{\equiv}}
\newcommand{\dv}{\ensuremath{\mathop{\raisebox{-2pt}{\vdots}}}}
\newcommand{\ndv}{\not \dv}

\newcounter{exnum}
\setcounter{exnum}{0}
\def\theexnum{{\emph{\arabic{exnum}}}}
\newcommand{\ex}{\refstepcounter{exnum}\vspace{2pt}\noindent \emph{Упражнение} \theexnum\emph{.}~}



\theoremstyle{plain}
\newtheorem{theorem}{Теорема}
\newtheorem{lemma}{Лемма}
\newtheorem{proposition}{Предложение}
\newtheorem{corollary}{Следствие}
\theoremstyle{definition}
\newtheorem{definition}{Определение}
\theoremstyle{remark}
\newtheorem{remark}{Замечание}
\newtheorem{example}{Пример}

\textheight=280mm %
\textwidth=180mm %
\oddsidemargin=-10.4mm%
\evensidemargin=-10.4mm %
\topmargin=-24.4mm


\begin{document}
	\pagestyle{empty}
    \name{}
	\head{группа 7.2}{Рыцари и не рыцари}
	\bigskip

\task Давным давно островитянин Дерб сказал своим друзьям: - Вчера мой сосед заявил мне, что он лжец! Кем является Дерб — рыцарем или лжецом?\\

\task Как-то раз встретились два островитянина и один сказал другому: «По крайней мере один из нас – лжец». История умалчивает, ответил ли ему на это что-либо собеседник. Тем не менее определите, кем являются оба.\\

\task Первый островитянин говорит второму: “Я лжец или ты рыцарь”. Кто из островитян кто?\\

\task В круг встали $10$  островитян. Каждый из них заявил, что следующие четверо после него — тролли. Сколько всего троллей среди них?\\
 
\task За круглым столом сидели $99$ человек, каждый из которых либо рыцарь, который всегда говорит правду, либо лжец, который всегда лжёт. Каждый из них сказал: «Хотя бы один из двух моих соседей — лжец.» Могло ли среди них быть ровно $60$ рыцарей?\\

\task Однажды в четверг после дождя между островитянами Тимом и Томом произошел следующий диалог: - Ты можешь сказать, что я рыцарь, - гордо заявил Тим. - Ты можешь сказать, что я лжец, - грустно ответил ему Том. Кем являются Тим и Том?\\

\task Все жители острова либо рыцари и говорят только правду, либо лжецы и всегда лгут. Путешественник встретил пятерых островитян. На его вопрос: «Сколько среди вас рыцарей?» первый ответил: «Ни одного!», а двое других ответили: «Один». Что ответили остальные?\\

\task За круглым столом сидят $100$ человек. Каждый из них либо рыцарь, либо лжец, либо чудак. Рыцарь всегда говорит правду, лжец всегда лжет. Чудак говорит правду, если слева от него сидит лжец; ложь, если слева от него сидит рыцарь; все что угодно, если слева от него сидит чудак. Каждый сказал: «Справа от меня сидит лжец». Сколько за столом лжецов? Перечислите все возможные ответы и докажите, что других нет. \\

Докажем это утверждение индукцией по количеству прямых.

{\bf База индукции.} Пусть на плоскости проведена всего одна прямая, которая делит эту плоскость на две части. Одну из этих частей покрасим в белый цвет, а другую — в черный. Тогда требование условия будет выполнено.

{\bf Шаг индукции.} Пусть на плоскости проведено k прямых, и части плоскости раскрашены в черный и белый цвета требуемым образом. Проведем (k + 1)-ую прямую. Все части плоскости по одну сторону от этой новой прямой перекрасим в противоположные цвета, а по другую сторону оставим все без изменений (см. рисунок). Убедитесь, что в этом случае части плоскости будут раскрашены требуемым образом.

\end{document}
