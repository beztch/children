\documentclass[a4paper,12pt]{extreport}
\usepackage[utf8]{inputenc}
\usepackage[russian]{babel}
\usepackage{ragged2e}
\usepackage[mag=1000,a4paper,left=1cm,right=1cm,top=1cm,bottom=1cm,noheadfoot]{geometry}
\usepackage{mathtext}
\usepackage{amsmath,amssymb,amsthm,amscd,amsfonts,graphicx,epsfig,textcomp,wrapfig}
\usepackage[dvips]{graphicx}
\graphicspath{{noiseimages/}}

\newcommand{\tab}{\hspace{10mm}}
\newcommand{\name}{
			\normalsize
			{
				\ \ \quad
                \mbox{} \hfil {\flushleft{5 школа}} \hfill {27.02.2024}
			%\quad
			}
			\vspace{5pt}\hrule
		}

\def\head#1#2{
	\begin{center}{
			\LARGE
			\bf #2
			
			{\normalsize \bf #1}
			\vspace{-10pt}
	}\end{center}
}

\newcommand{\del}{\mathop{\raisebox{-2pt}{\vdots}}}
\newcommand{\q}{}

\newcounter{tasknum}
\setcounter{tasknum}{0}
\def\thetasknum{{\textbf{\arabic{tasknum}}}}
\newcommand{\task}{\refstepcounter{tasknum}\vspace{2pt}\noindent \textbf{} \thetasknum\textbf{.}~}
\newcommand{\coff}{\refstepcounter{tasknum}\vspace{2pt}\noindent \textbf{Задача} \thetasknum *\textbf{.}~}


\newcommand{\eq}[1]{\underset{#1}{\equiv}}
\newcommand{\dv}{\ensuremath{\mathop{\raisebox{-2pt}{\vdots}}}}
\newcommand{\ndv}{\not \dv}

\newcounter{exnum}
\setcounter{exnum}{0}
\def\theexnum{{\emph{\arabic{exnum}}}}
\newcommand{\ex}{\refstepcounter{exnum}\vspace{2pt}\noindent \emph{Упражнение} \theexnum\emph{.}~}



\theoremstyle{plain}
\newtheorem{theorem}{Теорема}
\newtheorem{lemma}{Лемма}
\newtheorem{proposition}{Предложение}
\newtheorem{corollary}{Следствие}
\theoremstyle{definition}
\newtheorem{definition}{Определение}
\theoremstyle{remark}
\newtheorem{remark}{Замечание}
\newtheorem{example}{Пример}

\textheight=250mm %
\textwidth=180mm %
\oddsidemargin=-10.4mm%
\evensidemargin=-10.4mm %
\topmargin=-24.4mm


\begin{document}
	\pagestyle{empty}
    \name{}
	\head{6 класс}{Разнобой}
	\bigskip
	
\task Найдите все числа, равные удвоенной сумме своих цифр.\\

\task Трехзначное число начинается с цифры $4$. Если эту цифру перенести в конец числа, то получится число, составляющее три четверти исходного. Найти исходное число.\\

\task На сколько равных восьмиугольников можно разрезать квадрат размером $8 \times 8$? (Все разрезы должны проходить по линиям сетки.)
\\

\task Пусть принтер умеет печатать в неограниченном количестве купюры достоинством $m$ рублей или $n$ рублей ($m$ и $n$ взаимно просты).
\begin{itemize}
    \item[а)] Докажите, что ими можно без сдачи заплатить любую сумму, не меньшую $mn$ рублей.
    \item[б)] Докажите, что наибольшее число рублей, которое нельзя заплатить без сдачи, равно $mn − m − n$.
\end{itemize}

\task Саша загадал трехзначное число $\overline{abc}$ и сказал Пете сумму пяти (уже не обязательно трехзначных и различных) чисел $\overline{cab}, \overline{cba}, \overline{acb}, \overline{bca}, \overline{bac}$. Может ли Петя по этой информации однозначно восстановить число Саши?\\

\end{document}
