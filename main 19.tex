\documentclass[a4paper,12pt]{extreport}
\usepackage[utf8]{inputenc}
\usepackage[russian]{babel}
\usepackage{ragged2e}
\usepackage[mag=1000,a4paper,left=1cm,right=1cm,top=1cm,bottom=1cm,noheadfoot]{geometry}
\usepackage{mathtext}
\usepackage{amsmath,amssymb,amsthm,amscd,amsfonts,graphicx,epsfig,textcomp,wrapfig}
\usepackage[dvips]{graphicx}
\graphicspath{{noiseimages/}}

\newcommand{\tab}{\hspace{10mm}}
\newcommand{\name}{
			\normalsize
			{
				\ \ \quad
                \mbox{} \hfil {\flushleft{онлайн Взлет}} \hfill {30.09.2023}
			%\quad
			}
			\vspace{5pt}\hrule
		}

\def\head#1#2{
	\begin{center}{
			\LARGE
			\bf #2
			
			{\normalsize \bf #1}
			\vspace{-10pt}
	}\end{center}
}

\newcommand{\del}{\mathop{\raisebox{-2pt}{\vdots}}}
\newcommand{\q}{}

\newcounter{tasknum}
\setcounter{tasknum}{0}
\def\thetasknum{{\textbf{\arabic{tasknum}}}}
\newcommand{\task}{\refstepcounter{tasknum}\vspace{2pt}\noindent \textbf{} \thetasknum\textbf{.}~}
\newcommand{\coff}{\refstepcounter{tasknum}\vspace{2pt}\noindent \textbf{Задача} \thetasknum *\textbf{.}~}


\newcommand{\eq}[1]{\underset{#1}{\equiv}}
\newcommand{\dv}{\ensuremath{\mathop{\raisebox{-2pt}{\vdots}}}}
\newcommand{\ndv}{\not \dv}

\newcounter{exnum}
\setcounter{exnum}{0}
\def\theexnum{{\emph{\arabic{exnum}}}}
\newcommand{\ex}{\refstepcounter{exnum}\vspace{2pt}\noindent \emph{Упражнение} \theexnum\emph{.}~}



\theoremstyle{plain}
\newtheorem{theorem}{Теорема}
\newtheorem{lemma}{Лемма}
\newtheorem{proposition}{Предложение}
\newtheorem{corollary}{Следствие}
\theoremstyle{definition}
\newtheorem{definition}{Определение}
\theoremstyle{remark}
\newtheorem{remark}{Замечание}
\newtheorem{example}{Пример}

\textheight=250mm %
\textwidth=180mm %
\oddsidemargin=-10.4mm%
\evensidemargin=-10.4mm %
\topmargin=-24.4mm


\begin{document}
	\pagestyle{empty}
    \name{}
	\head{10 класс}{Изогональное сопряжение}
	\bigskip


Рассмотрим остроугольный треугольник $ABC$ (для неостроугольного треугольника рассуждения немного меняются, но в целом остаются аналогичными). Возьмём внутри него точку P. Пусть прямая $l_{a}$ симметрична прямой AP относительно биссектрисы угла $BAC$. Аналогично определим прямые $l_{b}$ и $l_{c}$.
\begin{theorem}
    Прямые $l_{a}$, $l_{b}$ и $l_{c}$ пересекаются в одной точке.
\end{theorem}
	
\task Пусть $Q$ — точка пересечения $l_{a}$ и $l_{b}$. Расмотрим описанную окружность $\omega$ треугольника $BQC$, а также вторую точку $T$ пересечения $l_{a}$ и $\omega$.
Докажите Теорему, показав, что $\angle PCA = \angle BCQ$.\\

\task Внутри треугольника $ABC$ отмечена точка $D$, а внутри треугольника $ABD$ — точка $E$. Оказалось, что $\angle BAE = \angle CBD, \ \angle EAD = \angle DBE, \ \angle DAC = \angle EBA$. Докажите, что точки $C, \ D, \ E$ лежат на одной прямой.\\

\task Изогонально сопряженные точки имеют общую педальную окружность.\\

\task Внутри остроугольного треугольника выбрана точка $Р$. $A_{1}B_{1}C_{1}$ – педальный треугольник точки $P$. Пусть точки $A_{2}, \ B_{2}, \ C_{2}$ – ортоцентры треугольников $AB_{1}C_{1}$ , $BA_{1}C_{1}$ и $CB_{1}A_{1}$ соответственно. Тогда прямые $A_{1}A_{2}, \ B_{1}B_{2}, \ C_{1}C_{2}$ пересекаются в одной точке.\\

\begin{theorem}
    Пусть $P$ – точка, лежащая внутри выпуклого четырехугольника $ABCD$. Для точки $P$ существует точка, изогонально сопряженная относительно $ABCD$, тогда и только тогда, когда $\angle APB + \angle CPD = 180^{\circ}$.
\end{theorem}

\task Докажите эквивалентность утверждений (можно сдавать следствия отдельно, например, $a \Rightarrow c$, но за это не более $1.5$ баллов. За целиком решенную задачу -- $2$ балла)
\begin{enumerate}
    \item[a)]  $\angle APB + \angle CPD = 180^{\circ}$
    \item[b)] Проекции точки $P$ на прямые $AB, \ BC, \ CD, \ DA$ лежат на одной окружности.
    \item[c)] Существует точка, изогонально сопряжённая точке P относительно ABCD.
\end{enumerate}

\task Выпуклый четырехугольник $ABCD$ описан около окружности с центром $I$. На отрезках $AI$ и $CI$ выбраны точки $P$ и $Q$ соответственно так,
что $\angle PBQ = \frac{1}{2} \angle ABC$. Докажите, что $\angle PDQ = \frac{1}{2} \angle AC$\\

\task  Пусть в остроугольном треугольнике $ABC$ точка $O$ – центр описанной окружности, $H$ – ортоцентр. Если $E$ и $F$ – точки пересечения срединного перпендикуляра к $BH$ с боковыми сторонами, то $OB$ – биссектриса
$\angle EOF$.\\

\end{document}
