\documentclass[a4paper,12pt]{extreport}
\usepackage[utf8]{inputenc}
\usepackage[russian]{babel}
\usepackage{ragged2e}
\usepackage[mag=1000,a4paper,left=1cm,right=1cm,top=1cm,bottom=1cm,noheadfoot]{geometry}
\usepackage{mathtext}
\usepackage{amsmath,amssymb,amsthm,amscd,amsfonts,graphicx,epsfig,textcomp,wrapfig}
\usepackage[dvips]{graphicx}
\graphicspath{{noiseimages/}}

\newcommand{\tab}{\hspace{10mm}}
\newcommand{\name}{
			\normalsize
			{
				\ \ \quad
                \mbox{} \hfil {\flushleft{Взлет}} \hfill {02.04.2024}
			%\quad
			}
			\vspace{5pt}\hrule
		}

\def\head#1#2{
	\begin{center}{
			\LARGE
			\bf #2
			
			{\normalsize \bf #1}
			\vspace{-10pt}
	}\end{center}
}

\newcommand{\del}{\mathop{\raisebox{-2pt}{\vdots}}}
\newcommand{\q}{}

\newcounter{tasknum}
\setcounter{tasknum}{0}
\def\thetasknum{{\textbf{\arabic{tasknum}}}}
\newcommand{\task}{\refstepcounter{tasknum}\vspace{2pt}\noindent \textbf{} \thetasknum\textbf{.}~}
\newcommand{\coff}{\refstepcounter{tasknum}\vspace{2pt}\noindent \textbf{Задача} \thetasknum *\textbf{.}~}


\newcommand{\eq}[1]{\underset{#1}{\equiv}}
\newcommand{\dv}{\ensuremath{\mathop{\raisebox{-2pt}{\vdots}}}}
\newcommand{\ndv}{\not \dv}

\newcounter{exnum}
\setcounter{exnum}{0}
\def\theexnum{{\emph{\arabic{exnum}}}}
\newcommand{\ex}{\refstepcounter{exnum}\vspace{2pt}\noindent \emph{Упражнение} \theexnum\emph{.}~}



\theoremstyle{plain}
\newtheorem{theorem}{Теорема}
\newtheorem{lemma}{Лемма}
\newtheorem{proposition}{Предложение}
\newtheorem{corollary}{Следствие}
\theoremstyle{definition}
\newtheorem{definition}{Определение}
\theoremstyle{remark}
\newtheorem{remark}{Замечание}
\newtheorem{example}{Пример}

\textheight=250mm %
\textwidth=180mm %
\oddsidemargin=-10.4mm%
\evensidemargin=-10.4mm %
\topmargin=-24.4mm


\begin{document}
	\pagestyle{empty}
    \name{}
	\head{9 класс}{Странные задачи}
	\bigskip

\task  Для действительных чисел $a$, $b$, $c$ выполнены следующие неравенства:
\[
    (a - b + c) (4 a - 2 b + c) < 0
\, , \qquad
    c (a - b + c) < 0
\, . \]
Докажите, что для этих чисел выполнено неравенство
$(a + b + c) (4 a + 2 b + c) \geq 0$.\\

\task На~множестве всех целых неотрицательных чисел определена операция $\ast$,
обладающая следующими свойствами:
$0 \ast y = y + 1$,
$(x + 1) \ast 0 = x \ast 1$,
$(x + 1) \ast (y + 1) = x \ast ((x + 1) \ast y)$.
\begin{itemize}
    \item[a)] Найдите $3 \ast 2017$.
    \item[b)] Найдите $4 \ast 2017$.
\end{itemize}

\task Докажите неравенство
\(
    \text{НОК}(a, b) \cdot \text{НОК}(b, c) \cdot \text{НОК}(a, c)
\geq
    \bigl( \text{НОК}(a, b, c) \bigr)^2
\).\\

\task На~множестве всех положительных чисел задана операция
$m \circ n = \frac{m + n}{m n + 4}$.
Найдите значение выражения
$(\ldots((2016 \circ 2015) \circ 2014) \circ \ldots \circ 2) \circ 1$.\\

\task На~множестве всех действительных чисел определена операция, обладающая
следующим свойством: $(x \ast y) \ast z = x + y + z$.
Докажите, что эта операция есть обычное сложение.\\

\task Последовательность натуральных чисел $a_{n}$ построена так, что для любых
$i \neq j$ выполнено свойство $\text{НОД}(a_i, a_j) = \text{НОД}(i, j)$.
Докажите, что при всех $i$ выполнено $a_i = i$.\\
 
\task Даны различные натуральные числа $a_1$, $a_2$, \ldots, $a_n$.
Докажите, что НОК всех чисел
\(
    b_{i}
=
    (a_{i} - a_{1}) (a_{i} - a_{2}) \ldots
    (a_{i} - a_{i-1}) (a_{i} - a_{i+1}) \ldots
    (a_{i} - a_{n})
\)
делится на~$(n-1)!$.\\

\task Докажите, что при любом натуральном $n$ можно указать $n$ последовательных натураль- ных чисел, среди которых нет ни одной степени натурального числа с показателем степени, большим $1$.\\

\end{document}
