\documentclass[a4paper,12pt]{extreport}
\usepackage[utf8]{inputenc}
\usepackage[russian]{babel}
\usepackage{ragged2e}
\usepackage[mag=1000,a4paper,left=1cm,right=1cm,top=1cm,bottom=1cm,noheadfoot]{geometry}
\usepackage{mathtext}
\usepackage{amsmath,amssymb,amsthm,amscd,amsfonts,graphicx,epsfig,textcomp,wrapfig}
\usepackage[dvips]{graphicx}
\graphicspath{{noiseimages/}}

\newcommand{\tab}{\hspace{10mm}}
\newcommand{\name}{
			\normalsize
			{
				\ \ \quad
                \mbox{} \hfil {\flushleft{Взлет}} \hfill {12.03.2024}
			%\quad
			}
			\vspace{5pt}\hrule
		}

\def\head#1#2{
	\begin{center}{
			\LARGE
			\bf #2
			
			{\normalsize \bf #1}
			\vspace{-10pt}
	}\end{center}
}

\newcommand{\del}{\mathop{\raisebox{-2pt}{\vdots}}}
\newcommand{\q}{}

\newcounter{tasknum}
\setcounter{tasknum}{0}
\def\thetasknum{{\textbf{\arabic{tasknum}}}}
\newcommand{\task}{\refstepcounter{tasknum}\vspace{2pt}\noindent \textbf{} \thetasknum\textbf{.}~}
\newcommand{\coff}{\refstepcounter{tasknum}\vspace{2pt}\noindent \textbf{Задача} \thetasknum *\textbf{.}~}


\newcommand{\eq}[1]{\underset{#1}{\equiv}}
\newcommand{\dv}{\ensuremath{\mathop{\raisebox{-2pt}{\vdots}}}}
\newcommand{\ndv}{\not \dv}

\newcounter{exnum}
\setcounter{exnum}{0}
\def\theexnum{{\emph{\arabic{exnum}}}}
\newcommand{\ex}{\refstepcounter{exnum}\vspace{2pt}\noindent \emph{Упражнение} \theexnum\emph{.}~}



\theoremstyle{plain}
\newtheorem{theorem}{Теорема}
\newtheorem{lemma}{Лемма}
\newtheorem{proposition}{Предложение}
\newtheorem{corollary}{Следствие}
\theoremstyle{definition}
\newtheorem{definition}{Определение}
\theoremstyle{remark}
\newtheorem{remark}{Замечание}
\newtheorem{example}{Пример}

\textheight=250mm %
\textwidth=180mm %
\oddsidemargin=-10.4mm%
\evensidemargin=-10.4mm %
\topmargin=-24.4mm


\begin{document}
	\pagestyle{empty}
    \name{}
	\head{9 класс}{Алгебра}
	\bigskip

\task Числа $a, b, c$ и $d$ удовлетворяют соотношению $a^2 + b^2 + (a + b)^2 = c^2 +d^2 +(c+d)^2$. Докажите, что $a^4 +b^4 +(a+b)^4 = c^4 +d^4 +(c+d)^4$.\\

\task Известно, что квадратные уравнения $ax^2 +bx+c = 0$ и $bx^2 + cx+a = 0 , \ (a,b$ и $c$ – отличные от нуля числа) имеют общий корень. Найдите его.\\

\task Найдите все такие числа $a$, что для любого натурального $n$ число $an(n + 2)(n + 4)$ будет целым.\\

\task Известно, что $x + 1$ – целое число. Докажите, что $x^n + \frac{1}{x^{n}}$ – также целое при любом целом $n$.\\

\task Существуют ли различные натуральные числа $a, b, c$, большие миллиарда, такие, что их произведение делится на любое из них, увеличенное на $2024$?\\

\task В таблицу $10 \times 10$ записаны положительные числа так, что в любой строке числа образуют арифметическую прогрессию (в порядке следования слева направо), а в любом столбце — геометрическую прогрессию (в порядке следования сверху вниз). Докажите, что знаменатели всех этих геометрических прогрессий равны.\\
 
\task При каком наименьшем натуральном $n$ существуют такие целые $a_{1}, a_{2} \dots a_{n}$, что уравнение $$x^{2} - 2(a_{1} + a_{2} + \dots + a_{n})^{2}x + (a_{1}^{4} + a_{2}^{4} + \dots + a_{n}^{4} + 1)$$ имеет по крайней мере один целый корень?\\

\task Изначально на доске был записан $n+1$ одночлен: $1, \ x, \ x^{2} \dots x^{n}$. Договорившись заранее, $k$ мальчиков каждую минуту одновременно вычисляли сумму каких-то двух многочленов, написанных на доске, и результат дописывали на доску. Через $m$ минут на доске, среди прочих, были написаны многочлены $S_{1} = 1 + x, \ S_{2} = 1 + x + x^{2},  \dots S_{n} = 1 + x + \dots + x^{n}$. Докажите, что $m \geq \frac{2n}{k+1}$.\\

\end{document}
