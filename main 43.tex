\documentclass[a4paper,12pt]{extreport}
\usepackage[utf8]{inputenc}
\usepackage[russian]{babel}
\usepackage{ragged2e}
\usepackage[mag=1000,a4paper,left=1cm,right=1cm,top=1cm,bottom=1cm,noheadfoot]{geometry}
\usepackage{mathtext}
\usepackage{amsmath,amssymb,amsthm,amscd,amsfonts,graphicx,epsfig,textcomp,wrapfig}
\usepackage[dvips]{graphicx}
\graphicspath{{noiseimages/}}

\newcommand{\tab}{\hspace{10mm}}
\newcommand{\name}{
			\normalsize
			{
				\ \ \quad
                \mbox{} \hfil {\flushleft{Взлет онлайн кружок}} \hfill {18.11.2023}
			%\quad
			}
			\vspace{5pt}\hrule
		}

\def\head#1#2{
	\begin{center}{
			\LARGE
			\bf #2
			
			{\normalsize \bf #1}
			\vspace{-10pt}
	}\end{center}
}

\newcommand{\del}{\mathop{\raisebox{-2pt}{\vdots}}}
\newcommand{\q}{}

\newcounter{tasknum}
\setcounter{tasknum}{0}
\def\thetasknum{{\textbf{\arabic{tasknum}}}}
\newcommand{\task}{\refstepcounter{tasknum}\vspace{2pt}\noindent \textbf{} \thetasknum\textbf{.}~}
\newcommand{\coff}{\refstepcounter{tasknum}\vspace{2pt}\noindent \textbf{Задача} \thetasknum *\textbf{.}~}


\newcommand{\eq}[1]{\underset{#1}{\equiv}}
\newcommand{\dv}{\ensuremath{\mathop{\raisebox{-2pt}{\vdots}}}}
\newcommand{\ndv}{\not \dv}

\newcounter{exnum}
\setcounter{exnum}{0}
\def\theexnum{{\emph{\arabic{exnum}}}}
\newcommand{\ex}{\refstepcounter{exnum}\vspace{2pt}\noindent \emph{Упражнение} \theexnum\emph{.}~}



\theoremstyle{plain}
\newtheorem{theorem}{Теорема}
\newtheorem{lemma}{Лемма}
\newtheorem{proposition}{Предложение}
\newtheorem{corollary}{Следствие}
\theoremstyle{definition}
\newtheorem{definition}{Определение}
\theoremstyle{remark}
\newtheorem{remark}{Замечание}
\newtheorem{example}{Пример}

\textheight=250mm %
\textwidth=180mm %
\oddsidemargin=-10.4mm%
\evensidemargin=-10.4mm %
\topmargin=-24.4mm


\begin{document}
	\pagestyle{empty}
    \name{}
	\head{8 класс}{Средняя линия}
	\bigskip

\task Докажите, что середины сторон произвольного четырёхугольника – вершины параллелограмма. 
Для каких четырёхугольников этот параллелограмм является прямоугольником, для каких – ромбом, для каких – квадратом?\\

\task Середины $E$ и $F$ параллельных сторон BC и AD параллелограмма $ABCD$ соединены с вершинами $D$ и $B$ соответственно. 
Докажите, что прямые $BF$ и $ED$ делят диагональ $AC$ на три равные части.\\

\task Точки $A_{1}$, $B_{1}$, $C_{1}$ – середины сторон соответственно $BC$, $AC$, $AB$ треугольника $ABC$. Пусть $E$ -- основание перпендикуляра из $B$ на $AC$. Докажите, что $BA_{1} = A_{1}E$ и $BC_{1} = C_{1}E$\\
 
\task Точки $A_{1}$, $B_{1}$, $C_{1}$ – середины сторон соответственно $BC$, $AC$, $AB$ треугольника $ABC$. Известно, что $A_{1}A$ и $B_{1}B$ – биссектрисы углов треугольника $A_{1}B_{1}C_{1}$. Найдите углы треугольника $ABC$.\\

\task Диагонали четырехугольника равны, а одна из его средних линий в два раза их меньше. Найдите угол между диагоналями.\\

\task Две противоположные стороны шестиугольника равны и параллельны. Докажите, что середины четырех оставшихся сторон образуют параллелограмм.\\

\task Средняя линия четырехугольника образует равные углы с его диагоналями. Докажите, что эти диагонали равны. 

\end{document}
