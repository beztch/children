\documentclass[a4paper,12pt]{extreport}
\usepackage[utf8]{inputenc}
\usepackage[russian]{babel}
\usepackage{ragged2e}
\usepackage[mag=1000,a4paper,left=1cm,right=1cm,top=1cm,bottom=1cm,noheadfoot]{geometry}
\usepackage{mathtext}
\usepackage{amsmath,amssymb,amsthm,amscd,amsfonts,graphicx,epsfig,textcomp,wrapfig}
\usepackage[dvips]{graphicx}
\graphicspath{{noiseimages/}}

\newcommand{\tab}{\hspace{10mm}}
\newcommand{\name}{
			\normalsize
			{
				\ \ \quad
                \mbox{} \hfil {\flushleft{5 школа сборы к ВСОШ}} \hfill {Казанцева Татьяна}
			%\quad
			}
			\vspace{5pt}\hrule
		}

\def\head#1#2{
	\begin{center}{
			\LARGE
			\bf #2
			
			{\normalsize \bf #1 $\bullet$ 14.04.2023}
			\vspace{-10pt}
	}\end{center}
}

\newcommand{\del}{\mathop{\raisebox{-2pt}{\vdots}}}
\newcommand{\q}{}

\newcounter{tasknum}
\setcounter{tasknum}{0}
\def\thetasknum{{\textbf{\arabic{tasknum}}}}
\newcommand{\task}{\refstepcounter{tasknum}\vspace{2pt}\noindent \textbf{} \thetasknum\textbf{.}~}
\newcommand{\coff}{\refstepcounter{tasknum}\vspace{2pt}\noindent \textbf{Задача} \thetasknum *\textbf{.}~}


\newcommand{\eq}[1]{\underset{#1}{\equiv}}
\newcommand{\dv}{\ensuremath{\mathop{\raisebox{-2pt}{\vdots}}}}
\newcommand{\ndv}{\not \dv}

\newcounter{exnum}
\setcounter{exnum}{0}
\def\theexnum{{\emph{\arabic{exnum}}}}
\newcommand{\ex}{\refstepcounter{exnum}\vspace{2pt}\noindent \emph{Упражнение} \theexnum\emph{.}~}



\theoremstyle{plain}
\newtheorem{theorem}{Теорема}
\newtheorem{lemma}{Лемма}
\newtheorem{proposition}{Предложение}
\newtheorem{corollary}{Следствие}
\theoremstyle{definition}
\newtheorem{definition}{Определение}
\theoremstyle{remark}
\newtheorem{remark}{Замечание}
\newtheorem{example}{Пример}

\textheight=250mm %
\textwidth=180mm %
\oddsidemargin=-10.4mm%
\evensidemargin=-10.4mm %
\topmargin=-24.4mm


\begin{document}
	\pagestyle{empty}
    \name{}
	\head{9 класс}{Разнобой}
	\bigskip

\task Приведенный квадратный трехчлен $P(x)$ таков, что $P(x)$ и $P(P(P(x)))$ имеют общий корень. Докажите, что $P(0) \cdot P(1) = 0$. \\

\task Петя загадал натуральное число $N$, Вася хочет его отгадать. Петя сначала говорит сумму цифр числа $N+1$, затем $N+2$, $N+3$ и так далее. Сможет ли Вася угадать Петино число? \\
	
\task Многочлены $F$ и $G$ таковы, что $$F(F(x)) > F(G(x)) > G(G(x))$$ для всех вещественных $x$. Докажите, что $F(x) > G(x)$ для всех вещественных $x$.\\

\task По кругу сидит $101$ человек, каждый из которых либо физик, либо астроном. Каждую минуту человек, оба соседа которого увлекаются противоположной наукой, меняет свой род деятельности на противоположный, а остальные не меняют. Докажите, что в какой-то момент смены прекратятся.\\ 

\task На сторонах $AB$, $BC$ и $AC$ треугольника $ABC$ выбраны точки $C_{1}$, $A_{1}$ и $B_{1}$ соответственно так, что $A_{1}C_{1} \parallel AC$ и $A_{1}B_{1} \parallel AB$. Прямая $B_{1}C_{1}$ пересекает окружность $(ABC)$ в точках $D$ и $E$. Докажите, что описанная окружность треугольника $A_{1}DE$ касается прямой $BC$.\\

\task Остроугольный треугольник $ABC$ вписан в окружность $\omega$. Касательные, проведенные к $\omega$ в точка $B$ и $C$ пересекаются в точке $P$. Точки $D$ и $E$ -- основания перпендикуляров из точки $P$ на $AB$ и $AC$ соответственно. Докажите, что высоты треугольника $ADE$ пересекаются в середине стороны $BC$.\\

\task Некто угнал старую полицейскую машину (скорость которой составляет a) $90\%$ б) $x\%$ от скорости новой) и поехал на ней в каком-то направлении по бесконечной в обе стороны дороге. Спустя какое-то время полицейские устроили погоню за ним на новой машине, но они не знают в какую он поехал сторону. Смогут ли они поймать гонщика?\\

\task На плоскости отмечено $N \geqslant 3$\ различных точек. Известно, что среди попарных расстояний между этими точками не более $n$ различных. Докажите, что $N \leqslant (n + 1)^{2}$.\\

\task В полном графе на $101$ вершине на каждом ребре поставлено число $+1$ или $−1$. Известно, что сумма чисел на всех ребрах по модулю меньше $150$. Докажите, что в графе найдется путь с нулевой суммой чисел на ребрах, проходящий по всем вершинам ровно по одному разу.\\

\end{document}
