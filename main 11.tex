\documentclass[a4paper,12pt]{extreport}
\usepackage[utf8]{inputenc}
\usepackage[russian]{babel}
\usepackage{ragged2e}
\usepackage[mag=1000,a4paper,left=1cm,right=1cm,top=1cm,bottom=1cm,noheadfoot]{geometry}
\usepackage{mathtext}
\usepackage{amsmath,amssymb,amsthm,amscd,amsfonts,graphicx,epsfig,textcomp,wrapfig}
\usepackage[dvips]{graphicx}
\graphicspath{{noiseimages/}}

\newcommand{\tab}{\hspace{10mm}}
\newcommand{\name}{
			\normalsize
			{
				\ \ \quad
                \mbox{} \hfil {\flushleft{5 лицей}} \hfill {05.10.2023}
			%\quad
			}
			\vspace{5pt}\hrule
		}

\def\head#1#2{
	\begin{center}{
			\LARGE
			\bf #2
			
			{\normalsize \bf #1}
			\vspace{-10pt}
	}\end{center}
}

\newcommand{\del}{\mathop{\raisebox{-2pt}{\vdots}}}
\newcommand{\q}{}

\newcounter{tasknum}
\setcounter{tasknum}{0}
\def\thetasknum{{\textbf{\arabic{tasknum}}}}
\newcommand{\task}{\refstepcounter{tasknum}\vspace{2pt}\noindent \textbf{} \thetasknum\textbf{.}~}
\newcommand{\coff}{\refstepcounter{tasknum}\vspace{2pt}\noindent \textbf{Задача} \thetasknum *\textbf{.}~}


\newcommand{\eq}[1]{\underset{#1}{\equiv}}
\newcommand{\dv}{\ensuremath{\mathop{\raisebox{-2pt}{\vdots}}}}
\newcommand{\ndv}{\not \dv}

\newcounter{exnum}
\setcounter{exnum}{0}
\def\theexnum{{\emph{\arabic{exnum}}}}
\newcommand{\ex}{\refstepcounter{exnum}\vspace{2pt}\noindent \emph{Упражнение} \theexnum\emph{.}~}



\theoremstyle{plain}
\newtheorem{theorem}{Теорема}
\newtheorem{lemma}{Лемма}
\newtheorem{proposition}{Предложение}
\newtheorem{corollary}{Следствие}
\theoremstyle{definition}
\newtheorem{definition}{Определение}
\theoremstyle{remark}
\newtheorem{remark}{Замечание}
\newtheorem{example}{Пример}

\textheight=250mm %
\textwidth=180mm %
\oddsidemargin=-10.4mm%
\evensidemargin=-10.4mm %
\topmargin=-24.4mm


\begin{document}
	\pagestyle{empty}
    \name{}
	\head{6 класс}{Движение}
	\bigskip
	
\task Из дома Юра вышел на $5$ минут позже Лены, но шёл со скоростью в два раза большей, чем она. Через какое время Юра догонит Лену?\\

\task Собаки Жучка и Полкан увидели друг друга во дворе и одновременно побежали навстречу друг другу. Через $5$ с, когда между ними оставалось $6$ м, Жучка испугалась и встала как вкопанная, а Полкан ещё через $2$ с подбежал к ней и поприветствовал. На каком расстоянии находились собаки, когда увидели друг друга, если Полкан бегает втрое быстрее Жучки?\\

\task Буратино и Пьеро бежали наперегонки. Пьеро весь путь бежал с одной и той же скоростью, а Буратино первую половину пути бежал вдвое быстрее, чем Пьеро, а вторую половину – вдвое медленней, чем Пьеро. Кто победил?\\

\task Бенедикт и Франциск красят забор. Каждый из них по отдельности может покрасить забор за $8$ часов. Забор начал красить Бенедикт, а спустя $2$ часа к нему присоединился Франциск. За сколько часов был покрашен весь забор?\\

\task Ванна заполняется холодной водой за $6$ минут $40$ секунд, горячей — за $8$ минут. Кроме того, если из полной ванны вынуть пробку, вода вытечет за $13$ минут $20$ секунд. Сколько времени понадобится, чтобы наполнить ванну полностью, при условии, что открыты оба крана, но ванна не заткнута пробкой?\\

\task Пони и ослик бегали с постоянными скоростями по кругу длиной $100$ м. Пони каждые две минуты обгонял ослика. Когда ослик вдвое увеличил скорость, он сам стал каждые две минуты обгонять пони. С какими скоростями бегали пони и ослик изначально?\\

\task Товарный поезд, отправившись из Москвы в $x$ часов $y$ минут, прибыл в Саратов в $y$ часов $z$ минут. Время в пути составило $z$ часов $x$ минут. 
Найдите все возможные значения $x$.\\

\end{document}
