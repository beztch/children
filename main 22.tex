\documentclass[a4paper,12pt]{extreport}
\usepackage[utf8]{inputenc}
\usepackage[russian]{babel}
\usepackage{ragged2e}
\usepackage[mag=1000,a4paper,left=1cm,right=1cm,top=1cm,bottom=1cm,noheadfoot]{geometry}
\usepackage{mathtext}
\usepackage{amsmath,amssymb,amsthm,amscd,amsfonts,graphicx,epsfig,textcomp,wrapfig}
\usepackage[dvips]{graphicx}
\graphicspath{{noiseimages/}}

\newcommand{\tab}{\hspace{10mm}}
\newcommand{\name}{
			\normalsize
			{
				\ \ \quad
                \mbox{} \hfil {\flushleft{Взлет}} \hfill {16.01.2024}
			%\quad
			}
			\vspace{5pt}\hrule
		}

\def\head#1#2{
	\begin{center}{
			\LARGE
			\bf #2
			
			{\normalsize \bf #1}
			\vspace{-10pt}
	}\end{center}
}

\newcommand{\del}{\mathop{\raisebox{-2pt}{\vdots}}}
\newcommand{\q}{}

\newcounter{tasknum}
\setcounter{tasknum}{0}
\def\thetasknum{{\textbf{\arabic{tasknum}}}}
\newcommand{\task}{\refstepcounter{tasknum}\vspace{2pt}\noindent \textbf{} \thetasknum\textbf{.}~}
\newcommand{\coff}{\refstepcounter{tasknum}\vspace{2pt}\noindent \textbf{Задача} \thetasknum *\textbf{.}~}


\newcommand{\eq}[1]{\underset{#1}{\equiv}}
\newcommand{\dv}{\ensuremath{\mathop{\raisebox{-2pt}{\vdots}}}}
\newcommand{\ndv}{\not \dv}

\newcounter{exnum}
\setcounter{exnum}{0}
\def\theexnum{{\emph{\arabic{exnum}}}}
\newcommand{\ex}{\refstepcounter{exnum}\vspace{2pt}\noindent \emph{Упражнение} \theexnum\emph{.}~}



\theoremstyle{plain}
\newtheorem{theorem}{Теорема}
\newtheorem{lemma}{Лемма}
\newtheorem{proposition}{Предложение}
\newtheorem{corollary}{Следствие}
\theoremstyle{definition}
\newtheorem{definition}{Определение}
\theoremstyle{remark}
\newtheorem{remark}{Замечание}
\newtheorem{example}{Пример}

\textheight=250mm %
\textwidth=180mm %
\oddsidemargin=-10.4mm%
\evensidemargin=-10.4mm %
\topmargin=-24.4mm


\begin{document}
	\pagestyle{empty}
    \name{}
	\head{8 класс}{Клеточки всякие3}
	\bigskip

\task Квадратная площадь размером $10 \times 10$ выложена квадратными плитами $1 \times 1$ двух цветов – белого и красного. Какое наибольшее число красных плит может оказаться среди них, если известно, что красные плиты не соприкасаются
(a) сторонами; (b) сторонами и уголками. \\
	
\task Клетки квадрата $9 \times 9$ окрашены в красный и синий цвета. Докажите, что найдется или клетка, у которой ровно два красных соседа по углу, или клетка, у которой ровно два синих соседа по углу (или и то, и другое).\\

\task Олег нарисовал пустую таблицу $50 \times 50$ и написал сверху от каждого столбца и слева от каждой строки по числу. Оказалось, что все $100$ написанных чисел различны, причём $50$ из них рациональные, а остальные $50$ — иррациональные. Затем в каждую клетку таблицы он записал произведение чисел, написанных около её строки и её столбца. Какое наибольшее количество чисел в этой таблице могли оказаться рациональными числами? \\

\task На изначально белой доске $2023 \times 2024$ двое по очереди закрашивают связные фигуры из $9$ клеток (перекрашивать уже закрашенные клетки запрещено). Кто выигрывает при правильной игре: начинающий или его соперник?
 \\

\task На доске $300 \times 300$ расставлены ладьи, они бьют всю доску. Известно, что каждая ладья бьет не более, чем одну другую ладью. При каком наименьшем $k$ в каждом квадрате $k \times k$ обязательно стоит хотя бы одна ладья?\\

\task Шахматную доску разбили на доминошки. Каждую из них требуется раскрасить каким-нибудь цветом так, чтобы любые две клетки, отстоящие на ход коня, были раскрашены в разные цвета. Какого наименьшего количества цветов заведомо хватит для этого? \\

\task Петя и Вася играют на доске $100 \times 100$. Изначально все клетки доски белые. Каждым своим ходом Петя красит в чёрный цвет одну или несколько белых клеток, стоящих подряд по диагонали. Каждым своим ходом Вася красит в черный цвет одну или несколько белых клеток, стоящих подряд по вертикали. Первый ход делает Петя. Проигрывает тот, кто не может сделать ход. Кто выигрывает при правильной игре? \\

\end{document}
