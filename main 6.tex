\documentclass[a4paper,12pt]{extreport}
\usepackage[utf8]{inputenc}
\usepackage[russian]{babel}
\usepackage{ragged2e}
\usepackage[mag=1000,a4paper,left=1cm,right=1cm,top=1cm,bottom=1cm,noheadfoot]{geometry}
\usepackage{mathtext}
\usepackage{amsmath,amssymb,amsthm,amscd,amsfonts,graphicx,epsfig,textcomp,wrapfig}
\usepackage[dvips]{graphicx}
\graphicspath{{noiseimages/}}

\newcommand{\tab}{\hspace{10mm}}
\newcommand{\name}{
			\normalsize
			{
				\ \ \quad
                \mbox{} \hfil {\flushleft{5 лицей}} \hfill {02.04.2024}
			%\quad
			}
			\vspace{5pt}\hrule
		}

\def\head#1#2{
	\begin{center}{
			\LARGE
			\bf #2
            \vspace{5pt}
			
			{\normalsize \bf #1}
			\vspace{-10pt}
	}\end{center}
}

\newcommand{\del}{\mathop{\raisebox{-2pt}{\vdots}}}
\newcommand{\q}{}

\newcounter{tasknum}
\setcounter{tasknum}{0}
\def\thetasknum{{\textbf{\arabic{tasknum}}}}
\newcommand{\task}{\refstepcounter{tasknum}\vspace{2pt}\noindent \textbf{} \thetasknum\textbf{.}~}
\newcommand{\coff}{\refstepcounter{tasknum}\vspace{2pt}\noindent \textbf{Задача} \thetasknum *\textbf{.}~}


\newcommand{\eq}[1]{\underset{#1}{\equiv}}
\newcommand{\dv}{\ensuremath{\mathop{\raisebox{-2pt}{\vdots}}}}
\newcommand{\ndv}{\not \dv}

\newcounter{exnum}
\setcounter{exnum}{0}
\def\theexnum{{\emph{\arabic{exnum}}}}
\newcommand{\ex}{\refstepcounter{exnum}\vspace{2pt}\noindent \emph{Упражнение} \theexnum\emph{.}~}



\theoremstyle{plain}
\newtheorem{theorem}{Теорема}
\newtheorem{lemma}{Лемма}
\newtheorem{proposition}{Предложение}
\newtheorem{corollary}{Следствие}
\theoremstyle{definition}
\newtheorem{definition}{Определение}
\theoremstyle{remark}
\newtheorem{remark}{Замечание}
\newtheorem{example}{Пример}

\textheight=250mm %
\textwidth=180mm %
\oddsidemargin=-10.4mm%
\evensidemargin=-10.4mm %
\topmargin=-24.4mm


\begin{document}
	\pagestyle{empty}
    \name{}
	\head{6 класс}{for (int i = 0; i < 7; ++i) \{solve(task[i]); \}}
	\bigskip

\task  Имеются двое песочных часов — на $7$ минут и на $11$ минут. Яйцо варится $15$ минут. Как отмерить это время при помощи имеющихся часов?\\

\task Каждая клетка доски $7 \times 7$ окрашена в черный или в белый цвет. Разрешается одновременно перекрасить все клетки
\begin{itemize}
    \item[(a)] некоторого столбца или некоторой строки в тот цвет, клеток которо- го в этом столбце или в этой строке до перекрашивания было больше.
    \item[(b)] любого прямоугольника 1 × 3 в тот цвет, клеток которого в этом пря- моугольнике до перекрашивания было больше.
\end{itemize}
Покажите, как такими действиями сделать все клетки одного цвета.

\task У Никиты есть калькулятор, который позволяет умножать число на $3$, прибавлять к числу $3$ или (если число делится на $3$ нацело) делить на $3$. Как на этом калькуляторе получить
\begin{itemize}
    \item[(a)] из числа $1$ любое натуральное число;
    \item[(b)] из любого натурального числа число $1$;
    \item[(c)] из любого натурального числа любое натуральное число?
\end{itemize}

\task Таня стоит на берегу речки. У неё есть два глиняных кувшина: один – на $5$ литров, а про второй Таня помнит лишь то, что он вмещает то ли $3$, то ли $4$ литра. Помогите Тане определить ёмкость второго кувшина. (Заглядывая в кувшин, нельзя понять, сколько в нём воды.)\\

\task Два мальчика живут в сёлах, между которыми по прямой трассе $90$ км, а их тётя — ровно посередине между ними. Тётя пригласила мальчиков в гости. У неё есть мопед, скорость которого $40$ км/ч, а с пассажиром — од- новременно: ребята выходят пешком со скоростью $5$ км/ч, а тётя выезжает на мопеде, по очереди подбирает мальчиков на трассе и подвозит их. Как им всем собраться у тёти не позднее, чем через $4$ часа после старта?\\

\task Король загадал одну из $36$ карт. За один ход мудрецу разрешается разложить карты на стопки с разным числом карт во всех стопках и узнать у короля, в какой из них лежит загаданная карта. Как мудрецу отгадать карту за два хода?\\
 
\task В одной деспотичной стране король созвал всех придворных мудрецов (их $10$ человек) и объявил им следующее: Завтра их всех построят в одну колонну и завяжут глаза, затем каждому на голову наденут колпак черного или белого цвета и снимут повязки. Каждый сможет видеть цвет колпака стоящих впереди него, но не может видеть свой колпак и колпаки тех, кто сзади. Каждому в колонне зададут вопрос: Какого цвета на тебе колпак? Если мудрец ответит правильно, его оставят в живых. Если неправильно, значит он недостоин быть мудрецом и его казнят. Какую стратегию надо избрать мудрецам, чтобы как можно больше из них остались в живых? На размышления и совещания им дается ровно одна ночь.\\

\end{document}
