\documentclass[a4paper,12pt]{extreport}
\usepackage[utf8]{inputenc}
\usepackage[russian]{babel}
\usepackage{ragged2e}
\usepackage[mag=1000,a4paper,left=1cm,right=1cm,top=1cm,bottom=1cm,noheadfoot]{geometry}
\usepackage{mathtext}
\usepackage{amsmath,amssymb,amsthm,amscd,amsfonts,graphicx,epsfig,textcomp,wrapfig}
\usepackage[dvips]{graphicx}
\graphicspath{{noiseimages/}}

\newcommand{\tab}{\hspace{10mm}}
\newcommand{\name}{
			\normalsize
			{
				\ \ \quad
                \mbox{} \hfil {\flushleft{Взлет МО}} \hfill {18.04.2023}
			%\quad
			}
			\vspace{5pt}\hrule
		}

\def\head#1#2{
	\begin{center}{
			\LARGE
			\bf #2
			
			{\normalsize \bf #1}
			\vspace{-10pt}
	}\end{center}
}

\newcommand{\del}{\mathop{\raisebox{-2pt}{\vdots}}}
\newcommand{\q}{}

\newcounter{tasknum}
\setcounter{tasknum}{0}
\def\thetasknum{{\textbf{\arabic{tasknum}}}}
\newcommand{\task}{\refstepcounter{tasknum}\vspace{2pt}\noindent \textbf{} \thetasknum\textbf{.}~}
\newcommand{\coff}{\refstepcounter{tasknum}\vspace{2pt}\noindent \textbf{Задача} \thetasknum *\textbf{.}~}


\newcommand{\eq}[1]{\underset{#1}{\equiv}}
\newcommand{\dv}{\ensuremath{\mathop{\raisebox{-2pt}{\vdots}}}}
\newcommand{\ndv}{\not \dv}

\newcounter{exnum}
\setcounter{exnum}{0}
\def\theexnum{{\emph{\arabic{exnum}}}}
\newcommand{\ex}{\refstepcounter{exnum}\vspace{2pt}\noindent \emph{Упражнение} \theexnum\emph{.}~}



\theoremstyle{plain}
\newtheorem{theorem}{Теорема}
\newtheorem{lemma}{Лемма}
\newtheorem{proposition}{Предложение}
\newtheorem{corollary}{Следствие}
\theoremstyle{definition}
\newtheorem{definition}{Определение}
\theoremstyle{remark}
\newtheorem{remark}{Замечание}
\newtheorem{example}{Пример}

\textheight=250mm %
\textwidth=180mm %
\oddsidemargin=-10.4mm%
\evensidemargin=-10.4mm %
\topmargin=-24.4mm


\begin{document}
	\pagestyle{empty}
    \name{}
	\head{8 класс}{Слепые алгоритмы. Добавка}
	\bigskip
	
\task В одном из расположенных в ряд $100$ окопов спрятался вражеский робот. В каком – неизвестно. Ваша задача – уничтожить робота. У вас есть пушка, которую можно навести на любой окоп и произвести выстрел. Если робот находился в этом окопе, задача выполнена. Если же робот был в другом окопе, то он, пока дым от выстрела не рассеялся, незаметно от нас обязательно перебегает в один из соседних окопов. Можно ли выполнить боевую задачу?\\

\task В таблице две строки и бесконечно много столбцов. В каждой строке и каждом столбце все числа различны. Докажите, что можно выбрать бесконечно много столбцов так, чтобы все числа в выбранных столбцах были различны.\\

\task Дана бесконечная клетчатая плоскость. Учительница и $30$ ее учеников играют в игру: за один ход можно покрасить в черный любой единичный отрезок по линиям сетки. Учительница побеждает, если после хода кого-то из игроков есть прямоугольник $1 \times 2$ в котором покрашены все границы, а отрезок внутри -- нет. Докажите, что учительница сможет выйграть независимо от ходов всех соперников. (Игроки ходят по-очереди. Сначала учительница, потом по порядку все ученики).\\

\end{document}
