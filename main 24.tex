\documentclass[a4paper,12pt]{extreport}
\usepackage[utf8]{inputenc}
\usepackage[russian]{babel}
\usepackage{ragged2e}
\usepackage[mag=1000,a4paper,left=1cm,right=1cm,top=1cm,bottom=1cm,noheadfoot]{geometry}
\usepackage{mathtext}
\usepackage{amsmath,amssymb,amsthm,amscd,amsfonts,graphicx,epsfig,textcomp,wrapfig}
\usepackage[dvips]{graphicx}
\graphicspath{{noiseimages/}}

\newcommand{\tab}{\hspace{10mm}}
\newcommand{\name}{
			\normalsize
			{
				\ \ \quad
                \mbox{} \hfil {\flushleft{Взлет МО}} \hfill {16.04.2023}
			%\quad
			}
			\vspace{5pt}\hrule
		}

\def\head#1#2{
	\begin{center}{
			\LARGE
			\bf #2
			
			{\normalsize \bf #1}
			\vspace{-10pt}
	}\end{center}
}

\newcommand{\del}{\mathop{\raisebox{-2pt}{\vdots}}}
\newcommand{\q}{}

\newcounter{tasknum}
\setcounter{tasknum}{0}
\def\thetasknum{{\textbf{\arabic{tasknum}}}}
\newcommand{\task}{\refstepcounter{tasknum}\vspace{2pt}\noindent \textbf{} \thetasknum\textbf{.}~}
\newcommand{\coff}{\refstepcounter{tasknum}\vspace{2pt}\noindent \textbf{Задача} \thetasknum *\textbf{.}~}


\newcommand{\eq}[1]{\underset{#1}{\equiv}}
\newcommand{\dv}{\ensuremath{\mathop{\raisebox{-2pt}{\vdots}}}}
\newcommand{\ndv}{\not \dv}

\newcounter{exnum}
\setcounter{exnum}{0}
\def\theexnum{{\emph{\arabic{exnum}}}}
\newcommand{\ex}{\refstepcounter{exnum}\vspace{2pt}\noindent \emph{Упражнение} \theexnum\emph{.}~}



\theoremstyle{plain}
\newtheorem{theorem}{Теорема}
\newtheorem{lemma}{Лемма}
\newtheorem{proposition}{Предложение}
\newtheorem{corollary}{Следствие}
\theoremstyle{definition}
\newtheorem{definition}{Определение}
\theoremstyle{remark}
\newtheorem{remark}{Замечание}
\newtheorem{example}{Пример}

\textheight=250mm %
\textwidth=180mm %
\oddsidemargin=-10.4mm%
\evensidemargin=-10.4mm %
\topmargin=-24.4mm


\begin{document}
	\pagestyle{empty}
    \name{}
	\head{7а класс}{Комбинаторика}
	\bigskip
	
\example Пусть у Маши было $7$ юбок, $14$ брюк и $5$ блузок. Она хочет выбрать наряд на вечер. Тогда понятно, что вариантов низов у нее $7 + 14 = 21$. Теперь предположим, что она все таки решила надеть юбку, но не решила какую, тогда у нее $7 \cdot 5 = 35$ вариантов наряда. Почему в первом случае мы складываем а во втором умножаем?\\

{\textbf{Правило суммы:}} Если некоторый объект $A$ можно выбрать $n$ способами, а другой объект $B$ можно выбрать $m$ способами, то выбор «либо $A$, либо $B$» можно осуществить $n + m$ способами.\\

{\textbf{Правило произведения:}} Если объект $A$ можно выбрать $n$ способами, а после каждого такого выбора другой объект $B$ можно выбрать (независимо от выбора объекта $A$) $m$ способами, то пары объектов $A$ и $B$ можно выбрать $n \cdot m$ способами.\\

\task В магазине «Все для чая» есть $5$ разных чашек, $3$ разных блюдца и $4$ разные чайные ложки.
\begin{itemize}
    \item[a)] Сколькими способами можно купить чашку с блюдцем?
    \item[б)] Сколькими способами можно купить комплект из чашки, блюдца и ложки?
    \item[в)] Сколькими способами можно купить два различных предмета?
\end{itemize}\\

\task Сколько существует пятизначных чисел? А сколько пятизначных, состоящих только из чётных цифр? А не содержащих цифру $3$ и тройку цифр $456$ подряд?\\

\task В классе $20$ человек. Тиебуется собрать команду на турнир матбоев, в которой должны быть тч-шник, комбинатор, геометр, алгебраист, отдыающий и капитан. Сколько есть вариантов собрать такую команду?\\

\example А если бы в прошлой задаче мы собирали команду ддя игры в вышибалы мы бы получили тот же ответ? Рассмотрим пример попроще: в компании 4 человека. Тогда вариантов выбрать из них капитана и заместителя $4 \cdot 3 = 12$, но выбрать пару всего $6$. Почему так происходит?\\

{\textbf{Число сочетаний из $\mathbf{n}$ предметов по $\mathbf{k}$}} -- число способов выбрать из $n$ предметов $k$, не учитывая порядок, в котором произведен выбор. И равно это число $$C_{n}^{k} = \frac{n \cdot (n - 1) \cdot ... \cdot (n - k + 1)}{k!} = \frac{n!}{k!(n - k)!}$$\\

\task У людоеда в подвале томятся 25 пленников.
\begin{itemize}
    \item[a)] Сколькими способами он может выбрать трех из них себе на завтрак, обед и ужин?
    \item[б)] А сколько есть способов выбрать троих, чтобы отпустить на свободу?
    \item[в)] Теперь он хочет отпустить сколько-то людей, но пока еще не понял, сколько именно и каких (возможно, $0$). Сколько способов у него это сделать?
\end{itemize}\\

\newpage

\task Сколько можно получить разных слов из букв ''МАТЕМАТИКА'', использованных в том же количестве?\\

\task Лёша принес в класс $36$ орехов и решил разделить их между собой, Максом и Борей. Сколько способов существует это сделать, если у каждого в итоге должен оказаться хотя бы один орех? А если не должен?

\task В шахматной доске удалили $4$ угловых клетки. Сколькими способами можно расставить на оставшиеся клетки 4 попарно не бьющие друг друга ладьи? (Способы, отличающиеся поворотом или переворотом, считаются различными.)\\

\task  Докажите, что уравнение $x! \cdot y!=z!$ имеет бесконечное число решений в натуральных числах, для которых $x, y \not = 1$.\\

\task В школе изучают $2n$ предметов. Все ученики учатся на $4$ и $5$. Никакие два ученика не учатся одинаково, ни про каких двух нельзя сказать, что один из них учится лучше другого. Доказать, что число учеников в школе не больше $C_{2n}^{n}$. (Мы считаем, что ученик А учится лучше ученика В, если у А оценки по всем предметам не ниже, чем у В, а по некоторым предметам – выше.)


\end{document}


