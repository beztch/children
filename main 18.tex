\documentclass[a4paper,12pt]{extreport}
\usepackage[utf8]{inputenc}
\usepackage[russian]{babel}
\usepackage{ragged2e}
\usepackage[mag=1000,a4paper,left=1cm,right=1cm,top=1cm,bottom=1cm,noheadfoot]{geometry}
\usepackage{mathtext}
\usepackage{amsmath,amssymb,amsthm,amscd,amsfonts,graphicx,epsfig,textcomp,wrapfig}
\usepackage[dvips]{graphicx}
\graphicspath{{noiseimages/}}

\newcommand{\tab}{\hspace{10mm}}
\newcommand{\name}{
			\normalsize
			{
				\ \ \quad
                \mbox{} \hfil {\flushleft{сборы взлет}} \hfill {02.11.2023}
			%\quad
			}
			\vspace{5pt}\hrule
		}

\def\head#1#2{
	\begin{center}{
			\LARGE
			\bf #2
			
			{\normalsize \bf #1}
			\vspace{-10pt}
	}\end{center}
}

\newcommand{\del}{\mathop{\raisebox{-2pt}{\vdots}}}
\newcommand{\q}{}

\newcounter{tasknum}
\setcounter{tasknum}{0}
\def\thetasknum{{\textbf{\arabic{tasknum}}}}
\newcommand{\task}{\refstepcounter{tasknum}\vspace{2pt}\noindent \textbf{} \thetasknum\textbf{.}~}
\newcommand{\coff}{\refstepcounter{tasknum}\vspace{2pt}\noindent \textbf{Задача} \thetasknum *\textbf{.}~}


\newcommand{\eq}[1]{\underset{#1}{\equiv}}
\newcommand{\dv}{\ensuremath{\mathop{\raisebox{-2pt}{\vdots}}}}
\newcommand{\ndv}{\not \dv}

\newcounter{exnum}
\setcounter{exnum}{0}
\def\theexnum{{\emph{\arabic{exnum}}}}
\newcommand{\ex}{\refstepcounter{exnum}\vspace{2pt}\noindent \emph{Упражнение} \theexnum\emph{.}~}



\theoremstyle{plain}
\newtheorem{theorem}{Теорема}
\newtheorem{lemma}{Лемма}
\newtheorem{proposition}{Предложение}
\newtheorem{corollary}{Следствие}
\theoremstyle{definition}
\newtheorem{definition}{Определение}
\theoremstyle{remark}
\newtheorem{remark}{Замечание}
\newtheorem{example}{Пример}

\textheight=290mm %
\textwidth=180mm %
\oddsidemargin=-10.4mm%
\evensidemargin=-10.4mm %
\topmargin=-24.4mm


\begin{document}
	\pagestyle{empty}
    \name{}
	\head{9 класс}{Ищем смысл}
	\bigskip


\task На доске написаны в порядке возрастания два натуральных числа $x$ и $y$ ($x \leq y$). Петя записывает на бумажке $x^{2}$ (квадрат первого числа), а затем заменяет числа на доске числами $x$ и $y – x$, записывая их в порядке возрастания. С новыми числами на доске он снова проделывает ту же операцию и так далее до тех пор, пока одно из чисел на доске не станет нулем. Чему будет в этот момент равна сумма чисел на Петиной бумажке?\\

\task На доске написаны три натуральных числа $a, \ b, \ c$. Петя записывает на бумажку произведение каких-нибудь двух из этих чисел, а на доске уменьшает третье число на $1$. С новыми числами на доске он снова проделывает ту же операцию, и так далее, пока одно из чисел на доске не станет нулем. Чему будет в этот момент равна сумма чисел на петиной бумажке?\\

\task  На столе лежала кучка серебряных монет. Каждым действием либо добавляли одну золотую монету и записывали количество серебряных монет на первый листок, либо убирали одну серебряную монету и записывали количество золотых монет на второй листок. В итоге на столе остались только золотые монеты. Докажите, что в этот момент сумма всех чисел на первом листке равнялась сумме всех чисел на втором. \\

\task  С начала учебного года Андрей записывал свои оценки по мате-
матике. Получая очередную оценку ($2, \ 3, \ 4$ или $5$), он называл её неожиданной, если до этого момента она встречалась реже каждой из всех остальных возможных оценок. (Например, если бы он получил с начала года подряд оценки $3, \ 4, \ 2, \ 5, \ 5, \ 5, \ 2, \ 3, \ 4, \ 3$, то неожиданными были бы первая пятёрка и вторая четвёрка.) За весь учебный год Андрей получил $40$ оценок — по $10$ пятерок, четверок, троек и двоек (не известно, в каком порядке). Можно ли точно сказать, сколько оценок были для него неожиданными?\\

\task Таблица $10 \times 10$ заполняется по правилам игры «Сапер»: в некоторые клетки ставят по мине, а в каждую из остальных клеток записывают количество мин в клетках, соседних с данной клеткой (по стороне или вершине). Может ли увеличиться сумма всех чисел в таблице, если все «старые» мины убрать, во все ранее свободные от мин клетки поставить мины, после чего заново записать числа по правилам?\\

\task Имеется три кучи камней. Сизиф таскает по одному камню из кучи в кучу. За каждое перетаскивание он получает от Зевса количество монет, равное разности числа камней в куче, в которую он кладет камень, и числа камней в куче, из которой он берет камень (сам перетаскиваемый камень при этом не учитывается). Если указанная разность отрицательна, то Сизиф возвращает Зевсу соответствующую сумму. (Если Сизиф не может расплатиться, то великодушный Зевс позволяет ему совершать перетаскивание в долг.) В некоторый момент оказалось, что все камни лежат в тех же кучах, в которых лежали первоначально. Каков наибольший суммарный заработок Сизифа на этот момент?\\

\task У каждого из $16$ детей было по 8 конфет. Каждую минуту один из детей платит в кассу столько рублей, у скольких детей конфет не меньше, чем у него, а затем съедает одну свою конфету. Сколько денег может оказаться в кассе, когда все конфеты будут съедены?\\

\end{document}
