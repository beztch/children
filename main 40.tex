\documentclass[a4paper,12pt]{extreport}
\usepackage[utf8]{inputenc}
\usepackage[russian]{babel}
\usepackage{ragged2e}
\usepackage[mag=1000,a4paper,left=1cm,right=1cm,top=1cm,bottom=1cm,noheadfoot]{geometry}
\usepackage{mathtext}
\usepackage{amsmath,amssymb,amsthm,amscd,amsfonts,graphicx,epsfig,textcomp,wrapfig}
\usepackage[dvips]{graphicx}
\graphicspath{{noiseimages/}}

\newcommand{\tab}{\hspace{10mm}}
\newcommand{\name}{
			\normalsize
			{
				\ \ \quad
                \mbox{} \hfil {\flushleft{Взлет МО}} \hfill {16.04.2023}
			%\quad
			}
			\vspace{5pt}\hrule
		}

\def\head#1#2{
	\begin{center}{
			\LARGE
			\bf #2
			
			{\normalsize \bf #1}
			\vspace{-10pt}
	}\end{center}
}

\newcommand{\del}{\mathop{\raisebox{-2pt}{\vdots}}}
\newcommand{\q}{}

\newcounter{tasknum}
\setcounter{tasknum}{0}
\def\thetasknum{{\textbf{\arabic{tasknum}}}}
\newcommand{\task}{\refstepcounter{tasknum}\vspace{2pt}\noindent \textbf{} \thetasknum\textbf{.}~}
\newcommand{\coff}{\refstepcounter{tasknum}\vspace{2pt}\noindent \textbf{Задача} \thetasknum *\textbf{.}~}


\newcommand{\eq}[1]{\underset{#1}{\equiv}}
\newcommand{\dv}{\ensuremath{\mathop{\raisebox{-2pt}{\vdots}}}}
\newcommand{\ndv}{\not \dv}

\newcounter{exnum}
\setcounter{exnum}{0}
\def\theexnum{{\emph{\arabic{exnum}}}}
\newcommand{\ex}{\refstepcounter{exnum}\vspace{2pt}\noindent \emph{Упражнение} \theexnum\emph{.}~}



\theoremstyle{plain}
\newtheorem{theorem}{Теорема}
\newtheorem{lemma}{Лемма}
\newtheorem{proposition}{Предложение}
\newtheorem{corollary}{Следствие}
\theoremstyle{definition}
\newtheorem{definition}{Определение}
\theoremstyle{remark}
\newtheorem{remark}{Замечание}
\newtheorem{example}{Пример}

\textheight=250mm %
\textwidth=180mm %
\oddsidemargin=-10.4mm%
\evensidemargin=-10.4mm %
\topmargin=-24.4mm


\begin{document}
	\pagestyle{empty}
    \name{}
	\head{8 класс}{Слепые алгоритмы}
	\bigskip

\task Двум узникам, сидящим в одиночных камерах, устраивают испытания. Каждому перед испытанием приносят кота – либо белого, либо чёрного. Узники должны угадать, какого цвета кот у его товарища. Если хотя бы один угадает, их выпустят на свободу, иначе – казнят. Перед испытанием узники могут договориться, а затем разойтись по камерам. Как им выйти на свободу?\\$$

\task Левша и невидимая блоха на плоскости играют, ходя по очереди. Очередным ходом Левша проводит прямую, а блоха совершает прыжок длины $1$, не пересекающий ни одной прямой. Если таких прыжков нет, блоха проигрывает. Может ли Левша выиграть, как бы не играла блоха?\\

\task Али-Баба подъехал к пещере с сокровищами. А перед входом хитрый замок.
Чтобы дверь в пещеру открылась, надо установить в одинаковое положение 4
переключателя, расположенные внутри вращающегося барабана за
отверстиями, расположенными на корпусе по кругу. Барабан вращается и
останавливается на очень короткое время. Хватает только, чтобы засунуть руки
в два из четырёх отверстий, нащупать там выключатели, и включить их (можно и
выключить, а можно и ничего не делать). Как только выключатели окажутся все включенными или все выключенными, двери откроются. Помогите Али-Бабе открыть дверь.\\

\task В классе $30$ человек, но не все между собой дружат, поэтому пришел школьный психолог, который может провести беседу с двумя враждующими, в результате чего они мирятся. Но если он случайно провел беседу с двумя друзьями, то они оба начинают враждовать с врагами другого (враг моего друга -- мой враг). Может ли школьный психолог, не зная, в каких отношениях состоят дети добиться идеально дружного класса?\\

\task Вдоль дороги стоит $128$ пронумерованных подряд столбов, как-то покрашенных в три цвета. Мэр столбов не видит. Он может назвать пару номеров, и если столбы разного цвета, их перекрасят в третий цвет, а если одинакового — то так и оставляют. В любом случае мэру ничего не докладывают. Всегда ли мэр может с помощью таких операций добиться, чтобы все столбы стали одинакового цвета?\\

\task Некто угнал старую полицейскую машину (скорость которой составляет a) $90\%$ б) $x\%$ (про $x$ полицейским известно только то, что $x < 100$) от скорости новой) и поехал на ней в каком-то направлении по бесконечной в обе стороны дороге. Спустя какое-то время полицейские устроили погоню за ним на новой машине, но они не знают в какую он поехал сторону. Смогут ли они поймать гонщика?\\

\task Назовём лабиринтом шахматную доску $8 \times 8$, на которой между некоторыми полями поставлены перегородки. По команде ВПРАВО ладья смещается на одно поле вправо или, если справа находится край доски или перегородка, остаётся на месте; аналогично выполняются команды ВЛЕВО, ВВЕРХ и ВНИЗ. Программист пишет программу — конечную последовательность указанных команд, и даёт её пользователю, после чего пользователь выбирает лабиринт и помещает в него ладью на любое поле. Верно ли, что программист может написать такую программу, что ладья обойдёт все доступные поля в лабиринте при любом выборе пользователя?\\

\end{document}
