\documentclass[a4paper,12pt]{extreport}
\usepackage[utf8]{inputenc}
\usepackage[russian]{babel}
\usepackage{ragged2e}
\usepackage[mag=1000,a4paper,left=1cm,right=1cm,top=1cm,bottom=1cm,noheadfoot]{geometry}
\usepackage{mathtext}
\usepackage{amsmath,amssymb,amsthm,amscd,amsfonts,graphicx,epsfig,textcomp,wrapfig}
\usepackage[dvips]{graphicx}
\graphicspath{{noiseimages/}}

\newcommand{\tab}{\hspace{10mm}}
\newcommand{\name}{
			\normalsize
			{
				\ \ \quad
                \mbox{} \hfil {\flushleft{Взлет}} \hfill {14.03.2024}
			%\quad
			}
			\vspace{5pt}\hrule
		}

\def\head#1#2{
	\begin{center}{
			\LARGE
			\bf #2
			
			{\normalsize \bf #1}
			\vspace{-10pt}
	}\end{center}
}

\newcommand{\del}{\mathop{\raisebox{-2pt}{\vdots}}}
\newcommand{\q}{}

\newcounter{tasknum}
\setcounter{tasknum}{0}
\def\thetasknum{{\textbf{\arabic{tasknum}}}}
\newcommand{\task}{\refstepcounter{tasknum}\vspace{2pt}\noindent \textbf{} \thetasknum\textbf{.}~}
\newcommand{\coff}{\refstepcounter{tasknum}\vspace{2pt}\noindent \textbf{Задача} \thetasknum *\textbf{.}~}


\newcommand{\eq}[1]{\underset{#1}{\equiv}}
\newcommand{\dv}{\ensuremath{\mathop{\raisebox{-2pt}{\vdots}}}}
\newcommand{\ndv}{\not \dv}

\newcounter{exnum}
\setcounter{exnum}{0}
\def\theexnum{{\emph{\arabic{exnum}}}}
\newcommand{\ex}{\refstepcounter{exnum}\vspace{2pt}\noindent \emph{Упражнение} \theexnum\emph{.}~}



\theoremstyle{plain}
\newtheorem{theorem}{Теорема}
\newtheorem{lemma}{Лемма}
\newtheorem{proposition}{Предложение}
\newtheorem{corollary}{Следствие}
\theoremstyle{definition}
\newtheorem{definition}{Определение}
\theoremstyle{remark}
\newtheorem{remark}{Замечание}
\newtheorem{example}{Пример}

\textheight=250mm %
\textwidth=180mm %
\oddsidemargin=-10.4mm%
\evensidemargin=-10.4mm %
\topmargin=-24.4mm


\begin{document}
	\pagestyle{empty}
    \name{}
	\head{10-11 класс}{Неравенства}
	\bigskip

\task Для положительных $a, b, c$ докажите неравенство $$\frac{a}{b +c} + \frac{b}{a + c} + \frac{c}{a + b} \geq \frac{3}{2}$$


\task Положительные числа $x, y$ и $z$ таковы, что $xyz \geq xy +yz + xz$. Докажите неравенство $$\sqrt{xyz} \geq \sqrt{x} +\sqrt{y} + \sqrt{z}$$.

\task Положительные числа $a, b, c$ удовлетворяют соотношению $ab + bc + ca = 3$. Докажите, что $$\sqrt{a + \frac{1}{a}} + \sqrt{b + \frac{1}{b}} + \sqrt{c + \frac{1}{c}} \geq 2 \left( \sqrt{a} + \sqrt{b} + \sqrt{c} \right)$$

\task Пусть $a, b$ и $c$ — положительные числа, такие, что $a + b + c = 3$. Докажите, что $$\frac{a}{1 + b^2} + \frac{b}{1 + 
c^2} + \frac{c}{1 + a^2} \geq \frac{3}{2}$$


\task Пусть $a, b$ и $c$ — положительные числа, такие, что $ab + ac + bc = 3$. Докажите, что $$ \frac{a}{2a + b^2} + \frac{b}{2b + c^2} + \frac{c}{2c + a^2} \leq 1$$


\task Даны числа $a, b, c$ не меньшие $1$. Докажите, что $$\frac{a + b + c}{4} \geq \frac{\sqrt{ab - 1}}{b + c} + \frac{\sqrt{bc - 1}}{c + a} + \frac{\sqrt{ca - 1}}{a + b}$$

\task Даны числа $x_{1} \dots x_{n} \in [1, 30]$. Докажите неравенство $$\sum_{i=1}^{n} \frac{7}{x_{i} + 6x_{i + 1}} \geq \sum_{i=1}^{n} \frac{6}{x_{i} + 5x_{i + 1}}$$

\task Даны числа $a, b, c > 0$. Докажите, что $$\sqrt{\frac{a}{b + c} + \frac{b}{c + a}} + \sqrt{\frac{b}{c + a} + \frac{c}{a + b}} + \sqrt{\frac{c}{a + b} + \frac{a}{b + c}} \geq 3$$
\end{document}
