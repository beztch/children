\documentclass[a4paper,12pt]{extreport}
\usepackage[utf8]{inputenc}
\usepackage[russian]{babel}
\usepackage{ragged2e}
\usepackage[mag=1000,a4paper,left=1cm,right=1cm,top=1cm,bottom=1cm,noheadfoot]{geometry}
\usepackage{mathtext}
\usepackage{amsmath,amssymb,amsthm,amscd,amsfonts,graphicx,epsfig,textcomp,wrapfig}
\usepackage[dvips]{graphicx}
\graphicspath{{noiseimages/}}

\newcommand{\tab}{\hspace{10mm}}
\newcommand{\name}{
			\normalsize
			{
				\ \ \quad
                \mbox{} \hfil {\flushleft{сборы взлет}} \hfill {14.10.2023}
			%\quad
			}
			\vspace{5pt}\hrule
		}

\def\head#1#2{
	\begin{center}{
			\LARGE
			\bf #2
			
			{\normalsize \bf #1}
			\vspace{-10pt}
	}\end{center}
}

\newcommand{\del}{\mathop{\raisebox{-2pt}{\vdots}}}
\newcommand{\q}{}

\newcounter{tasknum}
\setcounter{tasknum}{0}
\def\thetasknum{{\textbf{\arabic{tasknum}}}}
\newcommand{\task}{\refstepcounter{tasknum}\vspace{2pt}\noindent \textbf{} \thetasknum\textbf{.}~}
\newcommand{\coff}{\refstepcounter{tasknum}\vspace{2pt}\noindent \textbf{Задача} \thetasknum *\textbf{.}~}


\newcommand{\eq}[1]{\underset{#1}{\equiv}}
\newcommand{\dv}{\ensuremath{\mathop{\raisebox{-2pt}{\vdots}}}}
\newcommand{\ndv}{\not \dv}

\newcounter{exnum}
\setcounter{exnum}{0}
\def\theexnum{{\emph{\arabic{exnum}}}}
\newcommand{\ex}{\refstepcounter{exnum}\vspace{2pt}\noindent \emph{Упражнение} \theexnum\emph{.}~}



\theoremstyle{plain}
\newtheorem{theorem}{Теорема}
\newtheorem{lemma}{Лемма}
\newtheorem{proposition}{Предложение}
\newtheorem{corollary}{Следствие}
\theoremstyle{definition}
\newtheorem{definition}{Определение}
\theoremstyle{remark}
\newtheorem{remark}{Замечание}
\newtheorem{example}{Пример}

\textheight=250mm %
\textwidth=180mm %
\oddsidemargin=-10.4mm%
\evensidemargin=-10.4mm %
\topmargin=-24.4mm


\begin{document}
	\pagestyle{empty}
    \name{}
	\head{7а класс}{Игры разнобой}
	\bigskip
	
\task Двое игроков кладут одинаковые круглые монеты на прямоугольный стол; монеты могут свешиваться за край (но не должны падать) и не могут перекрываться. Кто не может положить монету, проигрывает. (Сдвигать ранее положенные монеты нельзя.)\\

\task На столе лежит куча из n конфет. Двое по очереди берут конфеты из кучи. За один ход можно взять $1, 2$ или $10$ камней. Проигрывает тот, кто не может сделать ход. При каких n начинающий выигрывает?\\

\task Два игрока забирают монеты из двух кучек, действуя по следующим правилам. Вначале первый игрок забирает себе из любой кучки несколько монет и перекладывает из этой кучки в другую такое же количество монет. Затем также поступает второй игрок (выбирая кучку, из которой он берет монеты, по своему усмотрению) и т.д. до тех пор, пока можно брать монеты по этим правилам. Игрок, взявший монеты последним, выигрывает. Кто из игроков имеет выигрышную стратегию, если кучки содержат (а) $20$ и $19$ монет (б) $20$ и $40$ монет\\

\task На доске написано число $123456789$. Два игрока ходят по очереди. За один ход можно отнять от числа на доске любую его ненулевую цифру и записать результат на доску, а старое число стереть.
а) Выигрывает тот, что получит $0$. Кто выигрывает при правильной игре? б) Проигрывает тот, кто получит ноль. Кто выигрывает при правильной игре?
 \\

\task Игра начинается с $60$. За ход разрешается уменьшить имеющееся число на любой из его делителей. Проигрывает тот, кто получит ноль\\

\end{document}
