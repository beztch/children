\documentclass[a4paper,12pt]{extreport}
\usepackage[utf8]{inputenc}
\usepackage[russian]{babel}
\usepackage{ragged2e}
\usepackage[mag=1000,a4paper,left=1cm,right=1cm,top=1cm,bottom=1cm,noheadfoot]{geometry}
\usepackage{mathtext}
\usepackage{amsmath,amssymb,amsthm,amscd,amsfonts,graphicx,epsfig,textcomp,wrapfig}
\usepackage[dvips]{graphicx}
\graphicspath{{noiseimages/}}

\newcommand{\tab}{\hspace{10mm}}
\newcommand{\name}{
			\normalsize
			{
				\ \ \quad
                \mbox{} \hfil {\flushleft{взлет онлайн}} \hfill {25.11.2023}
			%\quad
			}
			\vspace{5pt}\hrule
		}

\def\head#1#2{
	\begin{center}{
			\LARGE
			\bf #2
			
			{\normalsize \bf #1}
			\vspace{-10pt}
	}\end{center}
}

\newcommand{\del}{\mathop{\raisebox{-2pt}{\vdots}}}
\newcommand{\q}{}

\newcounter{tasknum}
\setcounter{tasknum}{0}
\def\thetasknum{{\textbf{\arabic{tasknum}}}}
\newcommand{\task}{\refstepcounter{tasknum}\vspace{2pt}\noindent \textbf{} \thetasknum\textbf{.}~}
\newcommand{\coff}{\refstepcounter{tasknum}\vspace{2pt}\noindent \textbf{Задача} \thetasknum *\textbf{.}~}


\newcommand{\eq}[1]{\underset{#1}{\equiv}}
\newcommand{\dv}{\ensuremath{\mathop{\raisebox{-2pt}{\vdots}}}}
\newcommand{\ndv}{\not \dv}

\newcounter{exnum}
\setcounter{exnum}{0}
\def\theexnum{{\emph{\arabic{exnum}}}}
\newcommand{\ex}{\refstepcounter{exnum}\vspace{2pt}\noindent \emph{Упражнение} \theexnum\emph{.}~}



\theoremstyle{plain}
\newtheorem{theorem}{Теорема}
\newtheorem{lemma}{Лемма}
\newtheorem{proposition}{Предложение}
\newtheorem{corollary}{Следствие}
\theoremstyle{definition}
\newtheorem{definition}{Определение}
\theoremstyle{remark}
\newtheorem{remark}{Замечание}
\newtheorem{example}{Пример}

\textheight=250mm %
\textwidth=180mm %
\oddsidemargin=-10.4mm%
\evensidemargin=-10.4mm %
\topmargin=-24.4mm


\begin{document}
	\pagestyle{empty}
    \name{}
	\head{8 класс}{Средняя линия 2}
	\bigskip
	
\task С помощью циркуля и линейки постройте треугольник, если даны середины трёх его сторон. \\

\task Разрежьте произвольный треугольник на три части, из которых можно сложить прямоугольник. \\

\task а) Одна из сторон треугольника равна $a$. Найдите длину отрезка, соединяющего середины медиан, проведённых к двум другим сторонам. б) Докажите, что отрезок, соединяющий середины диагоналей трапеции, равен полуразности её оснований. в) Докажите, что средняя линия трапеции (то есть отрезок, со- единяющий середины её боковых сторон) параллельна её основаниям и равна их полусумме.\\

\task Внутри произвольного угла взята точка $M$. С помощью циркуля и линейки проведите через точку $M$ прямую так, чтобы её отрезок, заключённый между сторонами угла, делился бы точкой $M$ пополам.
\\

\task На стороне $AC$ остроугольного треугольника $ABC$ выбрана точка $D$. Медиана $AM$ пересекает высоту $CH$ и отрезок $BD$ в точках $N$ и $K$ соответственно. 
Докажите, что если  $AK = BK$,  то  $AN = 2KM$.\\

\end{document}
