\documentclass[a4paper,12pt]{extreport}
\usepackage[utf8]{inputenc}
\usepackage[russian]{babel}
\usepackage{ragged2e}
\usepackage[mag=1000,a4paper,left=1cm,right=1cm,top=1cm,bottom=1cm,noheadfoot]{geometry}
\usepackage{mathtext}
\usepackage{amsmath,amssymb,amsthm,amscd,amsfonts,graphicx,epsfig,textcomp,wrapfig}
\usepackage[dvips]{graphicx}
\graphicspath{{noiseimages/}}

\newcommand{\tab}{\hspace{10mm}}
\newcommand{\name}{
			\normalsize
			{
				\ \ \quad
                \mbox{} \hfil {\flushleft{5 лицей}} \hfill {07.12.2023}
			%\quad
			}
			\vspace{5pt}\hrule
		}

\def\head#1#2{
	\begin{center}{
			\LARGE
			\bf #2
			
			{\normalsize \bf #1}
			\vspace{-10pt}
	}\end{center}
}

\newcommand{\del}{\mathop{\raisebox{-2pt}{\vdots}}}
\newcommand{\q}{}

\newcounter{tasknum}
\setcounter{tasknum}{0}
\def\thetasknum{{\textbf{\arabic{tasknum}}}}
\newcommand{\task}{\refstepcounter{tasknum}\vspace{2pt}\noindent \textbf{} \thetasknum\textbf{.}~}
\newcommand{\coff}{\refstepcounter{tasknum}\vspace{2pt}\noindent \textbf{Задача} \thetasknum *\textbf{.}~}


\newcommand{\eq}[1]{\underset{#1}{\equiv}}
\newcommand{\dv}{\ensuremath{\mathop{\raisebox{-2pt}{\vdots}}}}
\newcommand{\ndv}{\not \dv}

\newcounter{exnum}
\setcounter{exnum}{0}
\def\theexnum{{\emph{\arabic{exnum}}}}
\newcommand{\ex}{\refstepcounter{exnum}\vspace{2pt}\noindent \emph{Упражнение} \theexnum\emph{.}~}



\theoremstyle{plain}
\newtheorem{theorem}{Теорема}
\newtheorem{lemma}{Лемма}
\newtheorem{proposition}{Предложение}
\newtheorem{corollary}{Следствие}
\theoremstyle{definition}
\newtheorem{definition}{Определение}
\theoremstyle{remark}
\newtheorem{remark}{Замечание}
\newtheorem{example}{Пример}

\textheight=250mm %
\textwidth=180mm %
\oddsidemargin=-10.4mm%
\evensidemargin=-10.4mm %
\topmargin=-24.4mm


\begin{document}
	\pagestyle{empty}
    \name{}
	\head{6 класс}{Оценка $+$ пример}
	\bigskip
	
\task Какое наибольшее число трехклеточных уголков можно вырезать из клетча- того квадрата $8 \times 8$\\

\task Сумма нескольких различных натуральных слагаемых равна $50$. Какое наи- большее число слагаемых может быть?\\

\task На $22$ карточках написали числа от $1$ до $22$. Из этих карточек составили $11$ дробей. Какое наибольшее
их количество может после сокращения оказаться целыми числами?\\

\task В таблице $5 \times 5$ расставлены натуральные числа от $1$ до $25$. Оказалось, что в каждом столбце (сверху вниз) и в каждой строке (слева направо) числа идут в порядке возрастания. Найдите наименьшее возможное значение суммы чисел третьего столбца.\\

\task Том Сойер взялся покрасить очень длинный забор, соблюдая условия: любые две доски, между которыми ровно две, ровно три или ровно пять досок, должны быть окрашены в разные цвета. Какое наименьшее количество красок потребуется Тому для этой работы?\\

\task $8$ кузнецов должны подковать $10$ лошадей. Каждый кузнец тратит на одну подкову $5$ минут. Какое наименьшее время они должны потратить на работу? (Лошадь не может стоять на $2$ ногах)\\

\task Сумма десяти натуральных чисел равна $1001$. Какое наибольшее значение может принимать НОД (наибольший общий делитель) этих чисел?\\

\task Какое наибольшее количество кораблей $1 \times 2$ можно уложить на доске $10 \times 10$ без нарушения правил морского боя?\\

\end{document}
