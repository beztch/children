\documentclass[a4paper,12pt]{extreport}
\usepackage[utf8]{inputenc}
\usepackage[russian]{babel}
\usepackage{ragged2e}
\usepackage[mag=1000,a4paper,left=1cm,right=1cm,top=1cm,bottom=1cm,noheadfoot]{geometry}
\usepackage{mathtext}
\usepackage{amsmath,amssymb,amsthm,amscd,amsfonts,graphicx,epsfig,textcomp,wrapfig}
\usepackage[dvips]{graphicx}
\graphicspath{{noiseimages/}}

\newcommand{\tab}{\hspace{10mm}}
\newcommand{\name}{
			\normalsize
			{
				\ \ \quad
                \mbox{} \hfil {\flushleft{кружок 5 лицей}} \hfill {15.04.2023}
			%\quad
			}
			\vspace{5pt}\hrule
		}

\def\head#1#2{
	\begin{center}{
			\LARGE
			\bf #2
			
			{\normalsize \bf #1}
			\vspace{-10pt}
	}\end{center}
}

\newcommand{\del}{\mathop{\raisebox{-2pt}{\vdots}}}
\newcommand{\q}{}

\newcounter{tasknum}
\setcounter{tasknum}{0}
\def\thetasknum{{\textbf{\arabic{tasknum}}}}
\newcommand{\task}{\refstepcounter{tasknum}\vspace{2pt}\noindent \textbf{} \thetasknum\textbf{.}~}
\newcommand{\coff}{\refstepcounter{tasknum}\vspace{2pt}\noindent \textbf{Задача} \thetasknum *\textbf{.}~}


\newcommand{\eq}[1]{\underset{#1}{\equiv}}
\newcommand{\dv}{\ensuremath{\mathop{\raisebox{-2pt}{\vdots}}}}
\newcommand{\ndv}{\not \dv}

\newcounter{exnum}
\setcounter{exnum}{0}
\def\theexnum{{\emph{\arabic{exnum}}}}
\newcommand{\ex}{\refstepcounter{exnum}\vspace{2pt}\noindent \emph{Упражнение} \theexnum\emph{.}~}



\theoremstyle{plain}
\newtheorem{theorem}{Теорема}
\newtheorem{lemma}{Лемма}
\newtheorem{proposition}{Предложение}
\newtheorem{corollary}{Следствие}
\theoremstyle{definition}
\newtheorem{definition}{Определение}
\theoremstyle{remark}
\newtheorem{remark}{Замечание}
\newtheorem{example}{Пример}

\textheight=250mm %
\textwidth=180mm %
\oddsidemargin=-10.4mm%
\evensidemargin=-10.4mm %
\topmargin=-24.4mm


\begin{document}
	\pagestyle{empty}
    \name{}
	\head{5 класс}{Взвешивания}
	\bigskip
	
\task а) Перед вами $9$ монет, одна из них фальшивая (легче настоящих). Как за два взвешивания выяснить, какая монета фальшивая? б) Перед вами $81$ монета, одна из них фальшивая (легче настоящих). Сколько взвешиваний понадобится, чтобы выяснить, какая монета фальшивая? в) А если монет $10$, сколько взвешиваний понадобится?\\

\task Среди $4$ одинаковых на вид монет есть одна фальшивая (более тяжелая). За какое минимальное количество взвешиваний можно найти фальшивую монету?\\

\task Есть три монеты, одна из них фальшивая (но нам неизвестно, легче она или тяжелее, чем настоящие). Как выяснить, какая монета фальшивая? Обойдитесь как можно меньшим количеством взвешиваний.\\

\task Петя задумал какое-то число от $1$ до $16$, а Коля пытается его отгадать. Он задаёт Пете вопросы, на которые тот отвечает только «ДА» или «НЕТ». Как Коле отгадать число за четыре вопроса?\\

\task Лиса Алиса и Кот Базилио — фальшивомонетчики. Базилио делает монеты тяжелее настоящих, а Алиса — легче. У Буратино есть $15$ одинаковых по внешнему виду монет, но какая-то одна — фальшивая. Как двумя взвешиваниями на чашечных весах без гирь Буратино может определить, кто сделал фальшивую монету — Кот Базилио или Лиса Алиса? (Находить монету не надо.)\\

\task С помощью трех гирь продавец может взвесить любое целое число килограммов от $1$ до $10$ включительно. Какими гирями располагает продавец\\

\task Из пяти монет две фальшивые. Одна из фальшивых монет легче настоящей, а другая — на столько же тяжелее настоящей. Объясните, как за три взвешивания на чашечных весах без гирь найти обе фальшивые монеты.\\

\task Известно, что «медные» монеты достоинством в $1, 2, 3, 5$ коп. весят соответственно $1, 2, 3, 5$ г. Среди четырех «медных» монет (по одной каждого достоинства) есть одна бракованная, отличающаяся весом от нормальной. Как с помощью двух взвешиваний на чашечных весах без гирь определить бракованную монету?\\

\task  Золотоискатель добыл 9 кг песка. Ему надо отмерить $2$ кг песка с помощью весов с двумя чашами и одной гирей — $200$ г. Как это сделать за $3$ взвешивания? Если мы добиваемся равновесия путем пересыпания песка — это одно взвешивание. Есть пакеты для хранения песка.\\

\end{document}
