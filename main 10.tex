\documentclass[a4paper,12pt]{extreport}
\usepackage[utf8]{inputenc}
\usepackage[russian]{babel}
\usepackage{ragged2e}
\usepackage[mag=1000,a4paper,left=1cm,right=1cm,top=1cm,bottom=1cm,noheadfoot]{geometry}
\usepackage{mathtext}
\usepackage{amsmath,amssymb,amsthm,amscd,amsfonts,graphicx,epsfig,textcomp,wrapfig}
\usepackage[dvips]{graphicx}
\graphicspath{{noiseimages/}}

\newcommand{\tab}{\hspace{10mm}}
\newcommand{\name}{
			\normalsize
			{
				\ \ \quad
                \mbox{} \hfil {\flushleft{Взлет}} \hfill {06.12.2023}
			%\quad
			}
			\vspace{5pt}\hrule
		}

\def\head#1#2{
	\begin{center}{
			\LARGE
			\bf #2
			
			{\normalsize \bf #1}
			\vspace{-10pt}
	}\end{center}
}

\newcommand{\del}{\mathop{\raisebox{-2pt}{\vdots}}}
\newcommand{\q}{}

\newcounter{tasknum}
\setcounter{tasknum}{0}
\def\thetasknum{{\textbf{\arabic{tasknum}}}}
\newcommand{\task}{\refstepcounter{tasknum}\vspace{2pt}\noindent \textbf{} \thetasknum\textbf{.}~}
\newcommand{\coff}{\refstepcounter{tasknum}\vspace{2pt}\noindent \textbf{Задача} \thetasknum *\textbf{.}~}


\newcommand{\eq}[1]{\underset{#1}{\equiv}}
\newcommand{\dv}{\ensuremath{\mathop{\raisebox{-2pt}{\vdots}}}}
\newcommand{\ndv}{\not \dv}

\newcounter{exnum}
\setcounter{exnum}{0}
\def\theexnum{{\emph{\arabic{exnum}}}}
\newcommand{\ex}{\refstepcounter{exnum}\vspace{2pt}\noindent \emph{Упражнение} \theexnum\emph{.}~}



\theoremstyle{plain}
\newtheorem{theorem}{Теорема}
\newtheorem*{lemma}{Лемма}
\newtheorem{proposition}{Предложение}
\newtheorem{corollary}{Следствие}
\theoremstyle{definition}
\newtheorem{definition}{Определение}
\theoremstyle{remark}
\newtheorem{remark}{Замечание}
\newtheorem{example}{Пример}

\textheight=290mm %
\textwidth=180mm %
\oddsidemargin=-10.4mm%
\evensidemargin=-10.4mm %
\topmargin=-24.4mm


\begin{document}
	\pagestyle{empty}
    \name{}
	\head{9 класс}{Лемма Холла}
	\bigskip

\begin{lemma}[Холла]
    Есть $n$ юношей и несколько девушек. Известно, что каких бы $k$ юношей ни выбрать, число знакомых им совокупности девушек не меньше $k$. Тогда все юноши могут выбрать по невесте из числа своих знакомых.
\end{lemma}
	
\task  (Метод чередующихся цепей). Докажите лемму Холла по индукции, используя следующую конструкцию:

Предположим, некоторые пары заключили брак, но при этом один из юношей остался неженатым. Если кто-то из его знакомых девушек не замужем, то образуем пару. если же все замужем, то попробуем найти мужа, которого можно переженить так, чтобы невеста остановилась и т.д.
\\

\task (Индукция). Назовем множество из $k$ юношей критическим, если совокупное количество знакомых им девушек в точности ровно $k$.
\begin{itemize}
    \item [a)]  Предположим, что критическое множество юношей не содержит меньших критических подмножеств. Докажите, что никакая свадьба юноши из этого множества не испортит для остальных условие леммы Холла.
    \item [б)] Докажите, что если удалить критическое множество юношей, вместе с их знакомыми девушками, то для оставшихся будет выполнено условие леммы Холла.
    \item[в)] Докажите лемму Холла индукцией по числу вершин графа.
\end{itemize}
\\

\task Докажите, что если каждый юноша знает ровно $k$ девушек, а каждая девушка знает ровно $m$ юношей, причем $k \geq m$, то их можно переженить. \\

\task Лист бумаги карандашным рисунком разбит на $100$ областей одинаковой площади. С обратной стороны листа другим карандашным рисунком лист также разбит на $100$ областей одинаковой площади. Докажите, что лист можно проткнуть в сотне мест иголкой так, что каждая из $200$ нарисованных фигур будет иметь дырку внутри себя.\\

\task В каждой строке и каждом столбце шахматной доски стоит по три ладьи. Докажите, что можно выбрать 8 не бьющих друг друга ладей\\

\task Все вершины двудольного графа имеют степень k.
\begin{enumerate}
    \item [a)] Докажите, что в этом графе есть совершенное паросочетание.
    \item [б)] Докажите, что ребра этого графа можно раскрасить в k цветов правим образом (то есть так, чтобы никакие два одноцветных ребра не имели общей вершины).
\end{enumerate}
\\

\task Табло состоит из $2023$ лампочек. Двое играют в игру. Ход игрока состоит в том, что он изменяет состояние одной лампочки (т.е. включает и выключает ее). При этом нельзя повторять позицию, которая уже встречалась на табло. Проигрывает тот, кто не может сделать ход. Кто выигрывает при правильной игре?\\

\task Зритель пишет на доске слева направо $10$ цифр. Помощник фокусника закрывает одну из цифр карточкой. После этого входит фокусник и называет закрытую цифру. Докажите, что такой фокус действительно возможен.\\

\task Из $(n + 1)$-элементного множества выбраны $n$ подмножеств. Докажите, что некоторые элементы множества (хотя бы один) можно покрасить в один из двух цветов так, чтобы в каждом подмножестве были либо элементы обоих цветов, либо не было цветных элементов совсем.\\

\end{document}
