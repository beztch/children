\documentclass[a4paper,12pt]{extreport}
\usepackage[utf8]{inputenc}
\usepackage[russian]{babel}
\usepackage{ragged2e}
\usepackage[mag=1000,a4paper,left=1cm,right=1cm,top=1cm,bottom=1cm,noheadfoot]{geometry}
\usepackage{mathtext}
\usepackage{amsmath,amssymb,amsthm,amscd,amsfonts,graphicx,epsfig,textcomp,wrapfig}
\usepackage[dvips]{graphicx}
\graphicspath{{noiseimages/}}

\newcommand{\tab}{\hspace{10mm}}
\newcommand{\name}{
			\normalsize
			{
				\ \ \quad
                \mbox{} \hfil {\flushleft{5 школа}} \hfill {21.09.2023}
			%\quad
			}
			\vspace{5pt}\hrule
		}

\def\head#1#2{
	\begin{center}{
			\LARGE
			\bf #2
			
			{\normalsize \bf #1}
			\vspace{-10pt}
	}\end{center}
}

\newcommand{\del}{\mathop{\raisebox{-2pt}{\vdots}}}
\newcommand{\q}{}

\newcounter{tasknum}
\setcounter{tasknum}{0}
\def\thetasknum{{\textbf{\arabic{tasknum}}}}
\newcommand{\task}{\refstepcounter{tasknum}\vspace{2pt}\noindent \textbf{} \thetasknum\textbf{.}~}
\newcommand{\coff}{\refstepcounter{tasknum}\vspace{2pt}\noindent \textbf{Задача} \thetasknum *\textbf{.}~}


\newcommand{\eq}[1]{\underset{#1}{\equiv}}
\newcommand{\dv}{\ensuremath{\mathop{\raisebox{-2pt}{\vdots}}}}
\newcommand{\ndv}{\not \dv}

\newcounter{exnum}
\setcounter{exnum}{0}
\def\theexnum{{\emph{\arabic{exnum}}}}
\newcommand{\ex}{\refstepcounter{exnum}\vspace{2pt}\noindent \emph{Упражнение} \theexnum\emph{.}~}



\theoremstyle{plain}
\newtheorem{theorem}{Теорема}
\newtheorem{lemma}{Лемма}
\newtheorem{proposition}{Предложение}
\newtheorem{corollary}{Следствие}
\theoremstyle{definition}
\newtheorem{definition}{Определение}
\theoremstyle{remark}
\newtheorem{remark}{Замечание}
\newtheorem{example}{Пример}

\textheight=250mm %
\textwidth=180mm %
\oddsidemargin=-10.4mm%
\evensidemargin=-10.4mm %
\topmargin=-24.4mm


\begin{document}
	\pagestyle{empty}
    \name{}
	\head{6 класс}{Как я провел лето}
	\bigskip

\task Давным давно островитянин Костя сказал своим друзьям: - Вчера мой сосед заявил мне, что он лжец! Кем является Костя — рыцарем или лжецом?\\

\task Гирлянда, висящая вдоль школьного коридора, состоит из красных и синих лампочек. Рядом с каждой красной лампочкой обязательно есть синяя. Какое наибольшее число красных лампочек может быть в этой гирлянде, если всего лампочек $50$?\\
 
\task На метеорологической станции работает $19$ сотрудников. В соответствии с Кодексом Метеоролога в каждый ясный день на дежурство должны выйти два сотрудника, а в каждый пасмурный -- $5$ сотрудников. Выяснилось, что за август каждый из сотрудников выходил на дежурство ровно $5$ раз. Сколько было ясных дней в августе?\\

\task На доске записано $1012$ единиц и $1012$ нулей. За один ход Саша может либо стереть две одинаковые цифры и написать $0$, либо стереть две разные и написать $1$. Какая цифра останется через $2023$ хода?\\

\task Среди $6$ монет есть две фальшивые, которые весят меньше настоящих. Как за три взвешивания на чашечных весах найти обе фальшивые монеты?\\

\task Каких чисел среди первых $9999$ больше: тех, у которых сумма цифр $15$ или тех, у которых сумма цифр $21$?\\

\task Имеется $4$ предмета попарно различного веса. Как за $5$ взвешиваний на чашечных весах расположить их в порядке возрастания массы?\\

\task Федя и Никита играют в игру на шахматной доске. Сначала Федя ставит короля на любую клетку, а потом они по-очереди его двигают, начиная с Никиты. Проигрывает тот, кто пришел в клетку, в которой уже бывали до этого. У кого из игроков есть стратегия?\\

\end{document}
