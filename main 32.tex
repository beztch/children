\documentclass[a4paper,12pt]{extreport}
\usepackage[utf8]{inputenc}
\usepackage[russian]{babel}
\usepackage{ragged2e}
\usepackage[mag=1000,a4paper,left=1cm,right=1cm,top=1cm,bottom=1cm,noheadfoot]{geometry}
\usepackage{mathtext}
\usepackage{amsmath,amssymb,amsthm,amscd,amsfonts,graphicx,epsfig,textcomp,wrapfig}
\usepackage[dvips]{graphicx}
\graphicspath{{noiseimages/}}

\newcommand{\tab}{\hspace{10mm}}
\newcommand{\name}{
			\normalsize
			{
				\ \ \quad
                \mbox{} \hfil {\flushleft{5 школа}} \hfill {13.05.2023}
			%\quad
			}
			\vspace{5pt}\hrule
		}

\def\head#1#2{
	\begin{center}{
			\LARGE
			\bf #2
			
			{\normalsize \bf #1}
			\vspace{-10pt}
	}\end{center}
}

\newcommand{\del}{\mathop{\raisebox{-2pt}{\vdots}}}
\newcommand{\q}{}

\newcounter{tasknum}
\setcounter{tasknum}{0}
\def\thetasknum{{\textbf{\arabic{tasknum}}}}
\newcommand{\task}{\refstepcounter{tasknum}\vspace{2pt}\noindent \textbf{} \thetasknum\textbf{.}~}
\newcommand{\coff}{\refstepcounter{tasknum}\vspace{2pt}\noindent \textbf{Задача} \thetasknum *\textbf{.}~}


\newcommand{\eq}[1]{\underset{#1}{\equiv}}
\newcommand{\dv}{\ensuremath{\mathop{\raisebox{-2pt}{\vdots}}}}
\newcommand{\ndv}{\not \dv}

\newcounter{exnum}
\setcounter{exnum}{0}
\def\theexnum{{\emph{\arabic{exnum}}}}
\newcommand{\ex}{\refstepcounter{exnum}\vspace{2pt}\noindent \emph{Упражнение} \theexnum\emph{.}~}



\theoremstyle{plain}
\newtheorem{theorem}{Теорема}
\newtheorem{lemma}{Лемма}
\newtheorem{proposition}{Предложение}
\newtheorem{corollary}{Следствие}
\theoremstyle{definition}
\newtheorem{definition}{Определение}
\theoremstyle{remark}
\newtheorem{remark}{Замечание}
\newtheorem{example}{Пример}

\textheight=250mm %
\textwidth=180mm %
\oddsidemargin=-10.4mm%
\evensidemargin=-10.4mm %
\topmargin=-24.4mm


\begin{document}
	\pagestyle{empty}
    \name{}
	\head{5 класс}{Разумный перебор}
	\bigskip
	
\task Сколько существует двузначных чисел, у которых цифра десятков больше цифры единиц?\\

\task В слове 222122111121 каждая буква заменена своим номером в русском алфавите. Какое слово зашифровано?\\

\task В коробке лежат синие, красные и зеленые карандаши. Всего $20$ штук. Синих в $6$ раз больше, чем зеленых, красных меньше, чем синих. Сколько в коробке красных карандашей?\\

\task Кузнечик прыгает вдоль прямой вперёд на $80$ см или назад на $50$ см. Может ли он менее чем за $7$ прыжков удалиться от начальной точки ровно на $1$ м $70$ см?\\

\task После хоккейного матча Антон сказал, что он забил $3$ шайбы, а Илья только одну. Илья сказал, что он забил $4$ шайбы, а Серёжа целых $5$. Серёжа сказал, что он забил $6$ шайб, а Антон всего лишь две. Могло ли оказаться так, что втроём они забили $10$ шайб, если известно, что каждый из них один раз сказал правду, а другой раз солгал?\\

\task Четыре друга пришли с рыбалки. Каждые двое сосчитали суммы своих уловов. Получилось шесть чисел: $7$, $9$, $14$, $14$, $19$, $21$. Сможете ли Вы узнать, каковы были уловы?\\

\task Миша выписал подряд все числа месяца: $123456789101112$... и покрасил три дня (дни рождения своих друзей), никакие два из которых не идут подряд. Оказалось, что все непокрашенные участки состоят из одинакового количества цифр. Докажите, что первое число месяца покрашено.\\

\task На гранях кубика расставлены числа от $1$ до $6$. Кубик бросили два раза. В первый раз сумма чисел на четырёх боковых гранях оказалась равна $12$, во второй — $15$. Какое число написано на грани, противоположной той, где написана цифра $3$?\\

\task Студент за $5$ лет обучения сдал 31 экзамен. В каждом следующем году он сдавал больше экзаменов, чем в предыдущем, а на пятом курсе сдал втрое больше экзаменов, чем на первом курсе. Сколько экзаменов он сдал на четвертом курсе?\\

\end{document}
