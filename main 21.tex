\documentclass[a4paper,12pt]{extreport}
\usepackage[utf8]{inputenc}
\usepackage[russian]{babel}
\usepackage{ragged2e}
\usepackage[mag=1000,a4paper,left=1cm,right=1cm,top=1cm,bottom=1cm,noheadfoot]{geometry}
\usepackage{mathtext}
\usepackage{amsmath,amssymb,amsthm,amscd,amsfonts,graphicx,epsfig,textcomp,wrapfig}
\usepackage[dvips]{graphicx}
\graphicspath{{noiseimages/}}

\newcommand{\tab}{\hspace{10mm}}
\newcommand{\name}{
			\normalsize
			{
				\ \ \quad
                \mbox{} \hfil {\flushleft{взлет}} \hfill {18.01.2024}
			%\quad
			}
			\vspace{5pt}\hrule
		}

\def\head#1#2{
	\begin{center}{
			\LARGE
			\bf #2
			
			{\normalsize \bf #1}
			\vspace{-10pt}
	}\end{center}
}

\newcommand{\del}{\mathop{\raisebox{-2pt}{\vdots}}}
\newcommand{\q}{}

\newcounter{tasknum}
\setcounter{tasknum}{0}
\def\thetasknum{{\textbf{\arabic{tasknum}}}}
\newcommand{\task}{\refstepcounter{tasknum}\vspace{2pt}\noindent \textbf{} \thetasknum\textbf{.}~}
\newcommand{\coff}{\refstepcounter{tasknum}\vspace{2pt}\noindent \textbf{Задача} \thetasknum *\textbf{.}~}


\newcommand{\eq}[1]{\underset{#1}{\equiv}}
\newcommand{\dv}{\ensuremath{\mathop{\raisebox{-2pt}{\vdots}}}}
\newcommand{\ndv}{\not \dv}

\newcounter{exnum}
\setcounter{exnum}{0}
\def\theexnum{{\emph{\arabic{exnum}}}}
\newcommand{\ex}{\refstepcounter{exnum}\vspace{2pt}\noindent \emph{Упражнение} \theexnum\emph{.}~}



\theoremstyle{plain}
\newtheorem{theorem}{Теорема}
\newtheorem{lemma}{Лемма}
\newtheorem{proposition}{Предложение}
\newtheorem{corollary}{Следствие}
\theoremstyle{definition}
\newtheorem{definition}{Определение}
\theoremstyle{remark}
\newtheorem{remark}{Замечание}
\newtheorem{example}{Пример}

\textheight=250mm %
\textwidth=180mm %
\oddsidemargin=-10.4mm%
\evensidemargin=-10.4mm %
\topmargin=-24.4mm


\begin{document}
	\pagestyle{empty}
    \name{}
	\head{текст поменньше}{Текст}
	\bigskip
	
\task На стол положили (с перекрытиями) несколько одинаковых салфеток, имеющих форму правильного шестиугольника, причем у всех салфеток одна сторона параллельна одной и той же прямой. Всегда ли можно вбить в стол несколько гвоздей так, что все салфетка будут прибиты, причем каждая - только одним гвоздем?\\

\task Докажите, что найдется такое натуральное число $n > 1$, что произведение некоторых $n$ последовательных натуральных чисел равно произведению некоторых $n + 100$ последовательных натуральных чисел. \\

\task Сначала Саша прямолинейными разрезами, каждый из которых соединяет две точки на сторонах квадрата, делит квадрат со стороной $2$ на $2023$ частей. Затем Дима вырезает из каждой части по кругу. Докажите, что Дима всегда может добиться того, чтобы сумма радиусов этих кругов была не меньше $1$. \\

\task Найдите все натуральные числа, которые можно представить ввиде $\frac{xy+yz+zx}{}x + y + z$,где $х,у$ и $z$ —три различных натуральных числа. \\

\task  Постройте арифметическую прогрессию из $10000$ натуральных чисел, такую что суммы цифр этих чисел также образуют арифметическую прогрессию.\\

\task В ряд стоят $23$ коробочки с шариками, причём для каждого числа $n$ от $1$ до $23$ есть коробочка, в которой ровно $n$ шариков. За одну операцию можно переложить в любую коробочку еще столько же шариков, сколько в ней уже есть, из какой-нибудь другой коробочки, в которой шариков больше. Всегда ли можно такими операциями добиться, чтобы в первой коробочке оказался $1$ шарик, во второй – $2$ шарика, и так далее, в $23$-й – $23$ шарика?\\

\task В квадратной таблице $2021 \times 2023$ стоят натуральные числа. Можно выбрать любой столбец или любую строку в таблице и выполнить одно из следующих действий: 1) Прибавить к каждому выбранному числу $1$. 2) Разделить каждое из выбранных чисел на какое-нибудь натуральное число. Можно ли за несколько таких действий добиться того, чтобы каждое число в таблице было равно $1$?\\

\task В игре "Десант" две армии захватывают страну. Они ходят по очереди, каждым ходом занимая один из свободных городов. Первый свой город армия захватывает с воздуха, а каждым следующим ходом она может захватить любой город, соединённый дорогой с каким-нибудь уже занятым этой армией городом. Если таких городов нет, армия прекращает боевые действия (при этом, возможно, другая армия свои действия продолжает). Найдётся ли такая схема городов и дорог, что армия, ходящая второй, сможет захватить более половины всех городов, как бы ни действовала первая армия?
\\

\end{document}
