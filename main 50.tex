\documentclass[a4paper,12pt]{extreport}
\usepackage[utf8]{inputenc}
\usepackage[russian]{babel}
\usepackage{ragged2e}
\usepackage[mag=1000,a4paper,left=1cm,right=1cm,top=1cm,bottom=1cm,noheadfoot]{geometry}
\usepackage{mathtext}
\usepackage{amsmath,amssymb,amsthm,amscd,amsfonts,graphicx,epsfig,textcomp,wrapfig}
\usepackage[dvips]{graphicx}
\graphicspath{{noiseimages/}}

\newcommand{\tab}{\hspace{10mm}}
\newcommand{\name}{
			\normalsize
			{
				\ \ \quad
                \mbox{} \hfil {\flushleft{взлет МО}} \hfill {12.09.2023}
			%\quad
			}
			\vspace{5pt}\hrule
		}

\def\head#1#2{
	\begin{center}{
			\LARGE
			\bf #2
			
			{\normalsize \bf #1}
			\vspace{-10pt}
	}\end{center}
}

\newcommand{\del}{\mathop{\raisebox{-2pt}{\vdots}}}
\newcommand{\q}{}

\newcounter{tasknum}
\setcounter{tasknum}{0}
\def\thetasknum{{\textbf{\arabic{tasknum}}}}
\newcommand{\task}{\refstepcounter{tasknum}\vspace{2pt}\noindent \textbf{} \thetasknum\textbf{.}~}
\newcommand{\coff}{\refstepcounter{tasknum}\vspace{2pt}\noindent \textbf{Задача} \thetasknum *\textbf{.}~}


\newcommand{\eq}[1]{\underset{#1}{\equiv}}
\newcommand{\dv}{\ensuremath{\mathop{\raisebox{-2pt}{\vdots}}}}
\newcommand{\ndv}{\not \dv}

\newcounter{exnum}
\setcounter{exnum}{0}
\def\theexnum{{\emph{\arabic{exnum}}}}
\newcommand{\ex}{\refstepcounter{exnum}\vspace{2pt}\noindent \emph{Упражнение} \theexnum\emph{.}~}



\theoremstyle{plain}
\newtheorem{theorem}{Теорема}
\newtheorem{lemma}{Лемма}
\newtheorem{proposition}{Предложение}
\newtheorem{corollary}{Следствие}
\theoremstyle{definition}
\newtheorem{definition}{Определение}
\theoremstyle{remark}
\newtheorem{remark}{Замечание}
\newtheorem{example}{Пример}

\textheight=250mm %
\textwidth=180mm %
\oddsidemargin=-10.4mm%
\evensidemargin=-10.4mm %
\topmargin=-24.4mm


\begin{document}
	\pagestyle{empty}
    \name{}
	\head{8 класс}{Упорядочивание и усреднение}
	\bigskip
	
\task Лёша задумал четыре числа, а затем для каждой из шести пар чисел вы- числил их разность. Могли ли у него получиться разности $2, 3, 3, 4, 5, 6$?\\

\task В вершинах $2023$-угольника расставлены числа так, что каждое равно среднему арифметическому своих соседей. Докажите, что все они равны.\\

\task Пешеход шёл $2,5$ часа, причём за каждый промежуток времени в один час он проходил ровно $5$ км. Следует ли из этого, что его средняя скорость за всё время равна $5$ км/час?
\\

\task Есть $20$ карандашей, каждый из которых покрашен в один из $10$ цветов (по два карандаша каждого цвета). Они разложены в 10 коробок по $2$.  Докажите, можно выбрать по карандашу из каждой коробки так, чтобы все они были разных цветов.\\

\task На дискотеке $n$ юношей танцевали с $n$ девушками. В каждой паре юноша был выше девушки, но не более, чем на $10$ см. Докажите, что если поставить танцевать самого высокого юношу с самой высокой девушкой, второго по росту — со второй, и т. д., то по прежнему в каждой паре юноша будет выше девушки и опять же не более, чем на $10$ см.\\

\task Учитель провёл контрольную в двух классах $8$«A» и $8$«Б» и сосчитал рейтинг каждого класса как среднее арифметическое оценок, полученных учениками этого класса. „Вот если бы Вася учился не в $8$«Б», а в $8$«A», то рейтинг обоих классов был бы выше“ — начал мечтать учитель. Не перегрелся ли учитель, проверяя контрольные?\\

\task Две окружности равного радиуса разделены на 100 равных дуг каждая. На каждой окружности $50$ дуг покрашено в красный цвет, а $50$ — в синий. Докажите, что окружности можно совместить наложением так, чтобы цвета дуг совпали хотя бы в $50$ местах.\\

\end{document}
