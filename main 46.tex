\documentclass[a4paper,12pt]{extreport}
\usepackage[utf8]{inputenc}
\usepackage[russian]{babel}
\usepackage{ragged2e}
\usepackage[mag=1000,a4paper,left=1cm,right=1cm,top=1cm,bottom=1cm,noheadfoot]{geometry}
\usepackage{mathtext}
\usepackage{amsmath,amssymb,amsthm,amscd,amsfonts,graphicx,epsfig,textcomp,wrapfig}
\usepackage[dvips]{graphicx}
\graphicspath{{noiseimages/}}

\newcommand{\tab}{\hspace{10mm}}
\newcommand{\name}{
			\normalsize
			{
				\ \ \quad
                \mbox{} \hfil {\flushleft{Взлет}} \hfill {17.03.2024}
			%\quad
			}
			\vspace{5pt}\hrule
		}

\def\head#1#2{
	\begin{center}{
			\LARGE
			\bf #2
			
			{\normalsize \bf #1}
			\vspace{-10pt}
	}\end{center}
}

\newcommand{\del}{\mathop{\raisebox{-2pt}{\vdots}}}
\newcommand{\q}{}

\newcounter{tasknum}
\setcounter{tasknum}{0}
\def\thetasknum{{\textbf{\arabic{tasknum}}}}
\newcommand{\task}{\refstepcounter{tasknum}\vspace{2pt}\noindent \textbf{} \thetasknum\textbf{.}~}
\newcommand{\coff}{\refstepcounter{tasknum}\vspace{2pt}\noindent \textbf{Задача} \thetasknum *\textbf{.}~}


\newcommand{\eq}[1]{\underset{#1}{\equiv}}
\newcommand{\dv}{\ensuremath{\mathop{\raisebox{-2pt}{\vdots}}}}
\newcommand{\ndv}{\not \dv}

\newcounter{exnum}
\setcounter{exnum}{0}
\def\theexnum{{\emph{\arabic{exnum}}}}
\newcommand{\ex}{\refstepcounter{exnum}\vspace{2pt}\noindent \emph{Упражнение} \theexnum\emph{.}~}



\theoremstyle{plain}
\newtheorem{theorem}{Теорема}
\newtheorem{lemma}{Лемма}
\newtheorem{proposition}{Предложение}
\newtheorem{corollary}{Следствие}
\theoremstyle{definition}
\newtheorem*{definition}{Определение}
\theoremstyle{remark}
\newtheorem{remark}{Замечание}
\newtheorem{example}{Пример}

\textheight=250mm %
\textwidth=180mm %
\oddsidemargin=-10.4mm%
\evensidemargin=-10.4mm %
\topmargin=-24.4mm


\begin{document}
	\pagestyle{empty}
    \name{}
	\head{8 класс $\bullet$ 2 группа}{ТЧ степени вхождения}
	\bigskip

\task Докажите, что $n!$ не делится на $2^n.$\\

\task Даны натуральные числа $a$ и $b$, причем $a < 1000$. Докажите, что если $a^21$ делится на $b^10$, то $a^2$ делится на $b$.\\

\begin{definition}
    Степенью вхождения простого числа $p$ в натуральное число $n$ будем называть наибольшее такое $k$, что $n$ делится на $p^k$. Обозначать для краткости будем $\nu_p(n)$ (это греческая буква “ню”)
\end{definition}

\task Докажите следующие свойства:
\begin{itemize}
    \item[a)] $\nu_p(ab) = \nu_p(a) + \nu_p(b)$
    \item[b)] $\nu_p(a + b) \geq \min(\nu_p(a), \nu_p(b))$, причём если $\nu_p(a)  \not= \nu_p(b)$, то достигается равенство.
\end{itemize}


\task Натуральные числа $a$ и $b$ таковы, что сумма $\frac{b^2}{a} + \frac{a^2}{b}$ целая. Докажите, что оба слагаемых целые.\\

\task Натуральные числа $m, n$ таковы, что $m^2 + n^2 + m$ кратно $mn$. Докажите, что $m$ — квадрат натурального числа.\\


\task {\bf Формула Лежандра.} Докажите, что $$\nu_p(n!) = \bigg[ \frac{n}{p} \bigg] + \bigg[ \frac{n}{p^2} \bigg] + \bigg[ \frac{n}{p^3} \bigg] + \dots$$

\task Докажите, что для любых натуральных $a_1 , a_2 \dots a_k$ число $$\frac{(a_1 + a_2 + \dots a_k)!}{a_1! a_2! \dots a_k!}$$ целое.\\

\task Докажите, что наименьшее общее кратное чисел от $n$ до $2n + 1$
\begin{itemize}
    \item[a)] делится на $\frac{(2n + 1)!}{n!n!}$
    \item[b)] больше $4^n$.
\end{itemize}

\task Предположим, что среди чисел от $1$ до $2n+1$ ровно $k$ простых. Докажите, что $(2n + 1)^k > 4^n$.\\
\end{document}
