\documentclass[a4paper,12pt]{extreport}
\usepackage[utf8]{inputenc}
\usepackage[russian]{babel}
\usepackage{ragged2e}
\usepackage[mag=1000,a4paper,left=1cm,right=1cm,top=1cm,bottom=1cm,noheadfoot]{geometry}
\usepackage{mathtext}
\usepackage{amsmath,amssymb,amsthm,amscd,amsfonts,graphicx,epsfig,textcomp,wrapfig}
\usepackage[dvips]{graphicx}
\graphicspath{{noiseimages/}}

\newcommand{\tab}{\hspace{10mm}}
\newcommand{\name}{
			\normalsize
			{
				\ \ \quad
                \mbox{} \hfil {\flushleft{5 лицей}} \hfill {21.12.2023}
			%\quad
			}
			\vspace{5pt}\hrule
		}

\def\head#1#2{
	\begin{center}{
			\LARGE
			\bf #2
			
			{\normalsize \bf #1}
			\vspace{-10pt}
	}\end{center}
}

\newcommand{\del}{\mathop{\raisebox{-2pt}{\vdots}}}
\newcommand{\q}{}

\newcounter{tasknum}
\setcounter{tasknum}{0}
\def\thetasknum{{\textbf{\arabic{tasknum}}}}
\newcommand{\task}{\refstepcounter{tasknum}\vspace{2pt}\noindent \textbf{} \thetasknum\textbf{.}~}
\newcommand{\coff}{\refstepcounter{tasknum}\vspace{2pt}\noindent \textbf{Задача} \thetasknum *\textbf{.}~}


\newcommand{\eq}[1]{\underset{#1}{\equiv}}
\newcommand{\dv}{\ensuremath{\mathop{\raisebox{-2pt}{\vdots}}}}
\newcommand{\ndv}{\not \dv}

\newcounter{exnum}
\setcounter{exnum}{0}
\def\theexnum{{\emph{\arabic{exnum}}}}
\newcommand{\ex}{\refstepcounter{exnum}\vspace{2pt}\noindent \emph{Упражнение} \theexnum\emph{.}~}



\theoremstyle{plain}
\newtheorem{theorem}{Теорема}
\newtheorem{lemma}{Лемма}
\newtheorem{proposition}{Предложение}
\newtheorem{corollary}{Следствие}
\theoremstyle{definition}
\newtheorem{definition}{Определение}
\theoremstyle{remark}
\newtheorem{remark}{Замечание}
\newtheorem{example}{Пример}

\textheight=250mm %
\textwidth=180mm %
\oddsidemargin=-10.4mm%
\evensidemargin=-10.4mm %
\topmargin=-24.4mm


\begin{document}
	\pagestyle{empty}
    \name{}
	\head{6 класс}{Разнобой}
	\bigskip
	
\task В Стране дураков ходят монеты в $1, 2, 3, . . . , 19, 20$ сольдо (других нет). У Буратино была одна монета. Он купил мороженое и получил одну монету сдачи. Снова купил такое же мороженое и получил сдачу тремя монетами разного достоинства. Буратино хотел купить третье такое же мороженое, но денег не хватило. Сколько стоит мороженое?\\

\task Саша и Ваня родились $19$ марта. Каждый из них отмечает свой день рождения тортом со свечками по количеству исполнившихся ему лет. В тот год, когда они познакомились, у Саши на торте было столько же свечек, сколько у Вани сегодня. Известно, что суммарное количество свечек на четырёх тортах Вани и Саши (тогда и сегодня) равно $216$. Сколько лет исполнилось Ване сегодня?
\\

\task В большой квадратный зал привезли два квадратных ковра, сторона одного ковра вдвое больше стороны другого. Когда их положили в противоположные углы зала, они в два слоя накрыли $4$ м$^{2}$, а когда их положили в соседние углы, то $14$ м$^{2}$. Каковы размеры зала?\\

\task Петров забронировал квартиру в доме-новостройке, в котором пять одинако- вых подъездов. Изначально подъезды нумеровались слева направо, и квар- тира Петрова имела номер $636$. Потом застройщик поменял нумерацию на противоположную. Тогда квартира Петрова стала иметь номер $242$. Сколько квартир в доме? (Порядок нумерации квартир внутри подъезда не изменялся.)\\

\task Можно ли в равенстве $\frac{*}{*} + \frac{*}{*} + \frac{*}{*} + \frac{*}{*} + = *$ заменить звездочки цифрами от $1$ до $9$, взятыми по одному разу, так, чтобы равенство стало верным?\\

\task Четыре внешне одинаковые монетки весят $1, 2, 3$ и $4$ грамма. Можно ли за четыре взвешивания на чашечных весах без гирь узнать, какая из них сколько весит?\\

\task Два пирата, Билл и Джон, имея каждый по $74$ золотые монеты, решили сыг- рать в такую игру: они по очереди будут выкладывать на стол монеты, за один ход — одну, две или три, а выиграет тот, кто положит на стол сотую по счёту монету. Начинает Билл. Кто может выиграть в такой игре, независимо от того, как будет действовать соперник?\\

\task В каждой клетке доски размером $5 \times 5$ стоит крестик или нолик, причём никакие три крестика не стоят подряд ни по горизонтали, ни по вертикали, ни по диагонали. Какое наибольшее количество крестиков может быть на доске?\\

\end{document}
