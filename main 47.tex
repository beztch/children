\documentclass[a4paper,12pt]{extreport}
\usepackage[utf8]{inputenc}
\usepackage[russian]{babel}
\usepackage{ragged2e}
\usepackage[mag=1000,a4paper,left=1cm,right=1cm,top=1cm,bottom=1cm,noheadfoot]{geometry}
\usepackage{mathtext}
\usepackage{amsmath,amssymb,amsthm,amscd,amsfonts,graphicx,epsfig,textcomp,wrapfig}
\usepackage[dvips]{graphicx}
\graphicspath{{noiseimages/}}

\newcommand{\tab}{\hspace{10mm}}
\newcommand{\name}{
			\normalsize
			{
				\ \ \quad
                \mbox{} \hfil {\flushleft{взлет МО}} \hfill {04.09.2023}
			%\quad
			}
			\vspace{5pt}\hrule
		}

\def\head#1#2{
	\begin{center}{
			\LARGE
			\bf #2
			
			{\normalsize \bf #1}
			\vspace{-10pt}
	}\end{center}
}

\newcommand{\del}{\mathop{\raisebox{-2pt}{\vdots}}}
\newcommand{\q}{}

\newcounter{tasknum}
\setcounter{tasknum}{0}
\def\thetasknum{{\textbf{\arabic{tasknum}}}}
\newcommand{\task}{\refstepcounter{tasknum}\vspace{2pt}\noindent \textbf{} \thetasknum\textbf{.}~}
\newcommand{\coff}{\refstepcounter{tasknum}\vspace{2pt}\noindent \textbf{Задача} \thetasknum *\textbf{.}~}


\newcommand{\eq}[1]{\underset{#1}{\equiv}}
\newcommand{\dv}{\ensuremath{\mathop{\raisebox{-2pt}{\vdots}}}}
\newcommand{\ndv}{\not \dv}

\newcounter{exnum}
\setcounter{exnum}{0}
\def\theexnum{{\emph{\arabic{exnum}}}}
\newcommand{\ex}{\refstepcounter{exnum}\vspace{2pt}\noindent \emph{Упражнение} \theexnum\emph{.}~}



\theoremstyle{plain}
\newtheorem{theorem}{Теорема}
\newtheorem{lemma}{Лемма}
\newtheorem{proposition}{Предложение}
\newtheorem{corollary}{Следствие}
\theoremstyle{definition}
\newtheorem{definition}{Определение}
\theoremstyle{remark}
\newtheorem{remark}{Замечание}
\newtheorem{example}{Пример}

\textheight=250mm %
\textwidth=180mm %
\oddsidemargin=-10.4mm%
\evensidemargin=-10.4mm %
\topmargin=-24.4mm


\begin{document}
	\pagestyle{empty}
    \name{}
	\head{9 класс}{Название}
	\bigskip
	
\task Докажите, что существует $2023$-х значное число из цифр $2$ и $3$, делящееся на $2^{2023}$.\\

\task Составьте из всех $10$ цифр такое число, что число, образованное его первыми двумя цифрами делится на $2$, первыми тремя делится на $3$ ... первыми десятью на $10$. \\

\task Десять попарно различных ненулевых чисел таковы, что для каждых двух из них либо сумма этих чисел, либо их произведение -- рациональное число. Докажите, что квадраты всех чисел рациональны.\\

\task Верно ли, что найдутся $4$ последовательных натуральных числа, каждое из которых является точной степенью?\\

\task Нацдите все такие простые числа $p$ и $q$, что $pq - 555q$ и $pq + 555p$ явля/тся точными квадратами.\\


\task Обозначим через $S_{n}$ сумму первых $n$ простых чисел. Докажите, что для любого $k$ между $S_{k}$ и $S_{k + 1}$ найдется точный квадрат.\\

\task Докажите, что существует бесконечно много натуральных $n$ таких, что $n!$ делится на $n^{2} + 1$.\\

\task Докажите, что найдется $10^{100}$ подряд идущих натуральных чисел, каждое из которых имеет простой делитель, не являющийся делителем никакого из остальных.\\

\task На доску выписали все собственные делители некоторого натурального числа $n$, увеличенные на единицу. Найдите все такие $n$, что числа на доске являются всеми собственными делителями некоторого натурального числа $m$. (собственные делители числа $a > 1$ -- делители, отличные от $a$ и $1$).\\

\task Обозначим через $[k]!$ произведение $1 \cdot 11 \cdot 111 \cdot ... \cdot 11...1$ ($k$ сомножителей). Докажите, что $[n + m]!$ делится на $[n]![m]!$.\\

\end{document}
