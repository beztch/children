\documentclass[a4paper,12pt]{extreport}
\usepackage[utf8]{inputenc}
\usepackage[russian]{babel}
\usepackage{ragged2e}
\usepackage[mag=1000,a4paper,left=1cm,right=1cm,top=1cm,bottom=1cm,noheadfoot]{geometry}
\usepackage{mathtext}
\usepackage{amsmath,amssymb,amsthm,amscd,amsfonts,graphicx,epsfig,textcomp,wrapfig}
\usepackage[dvips]{graphicx}
\graphicspath{{noiseimages/}}

\newcommand{\tab}{\hspace{10mm}}
\newcommand{\name}{
			\normalsize
			{
				\ \ \quad
                \mbox{} \hfil {\flushleft{сборы 5 лицей}} \hfill {27.09.2024}
			%\quad
			}
			\vspace{5pt}\hrule
		}

\def\head#1#2{
	\begin{center}{
			\LARGE
			\bf #2
			
			{\normalsize \bf #1}
			\vspace{-10pt}
	}\end{center}
}

\newcommand{\del}{\mathop{\raisebox{-2pt}{\vdots}}}
\newcommand{\q}{}

\newcounter{tasknum}
\setcounter{tasknum}{0}
\def\thetasknum{{\textbf{\arabic{tasknum}}}}
\newcommand{\task}{\refstepcounter{tasknum}\vspace{2pt}\noindent \textbf{} \thetasknum\textbf{.}~}
\newcommand{\coff}{\refstepcounter{tasknum}\vspace{2pt}\noindent \textbf{Задача} \thetasknum *\textbf{.}~}


\newcommand{\eq}[1]{\underset{#1}{\equiv}}
\newcommand{\dv}{\ensuremath{\mathop{\raisebox{-2pt}{\vdots}}}}
\newcommand{\ndv}{\not \dv}

\newcounter{exnum}
\setcounter{exnum}{0}
\def\theexnum{{\emph{\arabic{exnum}}}}
\newcommand{\ex}{\refstepcounter{exnum}\vspace{2pt}\noindent \emph{Упражнение} \theexnum\emph{.}~}



\theoremstyle{plain}
\newtheorem{theorem}{Теорема}
\newtheorem{lemma}{Лемма}
\newtheorem{proposition}{Предложение}
\newtheorem{corollary}{Следствие}
\theoremstyle{definition}
\newtheorem{definition}{Определение}
\theoremstyle{remark}
\newtheorem{remark}{Замечание}
\newtheorem{example}{Пример}

\textheight=250mm %
\textwidth=180mm %
\oddsidemargin=-10.4mm%
\evensidemargin=-10.4mm %
\topmargin=-24.4mm


\begin{document}
	\pagestyle{empty}
    \name{}
	\head{10-11 класс}{АТЧ}
	\bigskip

\task Верно ли, что найдутся $4$ последовательных натуральных числа, каждое из которых является точной степенью?\\

\task При каких натуральных $n$ число $n^4 + 1$ делится на $n^2 + n + 1$?\\

\task Существует ли натуральное число $N$ такое, что если приписать его к самому себе справа, то полученное число окажется точным квадратом? \\
	
\task Докажите, что существует $2023$-х значное число из цифр $2$ и $3$, делящееся на $2^{2023}$.\\

\task Можно ли вычеркнуть из произведения  $1! \cdot 2! \cdot 3! \cdot \dots \cdot 100!$  один из факториалов так, чтобы произведение оставшихся было квадратом целого числа?\\

\task Для положительных $a, b, c$ докажите неравенство $a + b + c \leq \frac{a^2 + b^2}{2c} + \frac{b^2 + c^2}{2a} + \frac{c^2 + a^2}{2b} \leq \frac{a^3}{bc} + \frac{b^3}{ac} + \frac{c^3}{ab}$.\\

\task Пусть $\tau(n)$ – количество положительных делителей натурального числа $n$. Решите уравнение  $a = 2 \tau (a)$.\\

\task Десять попарно различных ненулевых чисел таковы, что для каждых двух из них либо сумма этих чисел, либо их произведение -- рациональное число. Докажите, что квадраты всех чисел рациональны.\\

\task Пусть $x, y$ натуральные числа, большие $1$. Оказалось, что $x^2 + y^2 - 1$ делится на $x + y - 1$. Докажите, что $x + y - 1$ - составное.\\

\task Натуральные числа $a$ и $b$ таковы, что $a^3 + a^2 +b$. точный куб. Может ли число $b^2 + a$ быть точным квадратом?\\

\task По кругу стоят $10^{1000}$ натуральных чисел. Между каждыми двумя соседними числами записали их наименьшее общее кратное. 
Могут ли эти наименьшие общие кратные образовать $10^{1000}$ последовательных чисел (расположенных в каком-то порядке)?\\

\end{document}
