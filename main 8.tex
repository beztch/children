\documentclass[a4paper,12pt]{extreport}
\usepackage[utf8]{inputenc}
\usepackage[russian]{babel}
\usepackage{ragged2e}
\usepackage[mag=1000,a4paper,left=1cm,right=1cm,top=1cm,bottom=1cm,noheadfoot]{geometry}
\usepackage{mathtext}
\usepackage{amsmath,amssymb,amsthm,amscd,amsfonts,graphicx,epsfig,textcomp,wrapfig}
\usepackage[dvips]{graphicx}
\graphicspath{{noiseimages/}}

\newcommand{\tab}{\hspace{10mm}}
\newcommand{\name}{
			\normalsize
			{
				\ \ \quad
                \mbox{} \hfil {\flushleft{5 лицей}} \hfill {16.11.2023}
			%\quad
			}
			\vspace{5pt}\hrule
		}

\def\head#1#2{
	\begin{center}{
			\LARGE
			\bf #2
			
			{\normalsize \bf #1}
			\vspace{-10pt}
	}\end{center}
}

\newcommand{\del}{\mathop{\raisebox{-2pt}{\vdots}}}
\newcommand{\q}{}

\newcounter{tasknum}
\setcounter{tasknum}{0}
\def\thetasknum{{\textbf{\arabic{tasknum}}}}
\newcommand{\task}{\refstepcounter{tasknum}\vspace{2pt}\noindent \textbf{} \thetasknum\textbf{.}~}
\newcommand{\coff}{\refstepcounter{tasknum}\vspace{2pt}\noindent \textbf{Задача} \thetasknum *\textbf{.}~}


\newcommand{\eq}[1]{\underset{#1}{\equiv}}
\newcommand{\dv}{\ensuremath{\mathop{\raisebox{-2pt}{\vdots}}}}
\newcommand{\ndv}{\not \dv}

\newcounter{exnum}
\setcounter{exnum}{0}
\def\theexnum{{\emph{\arabic{exnum}}}}
\newcommand{\ex}{\refstepcounter{exnum}\vspace{2pt}\noindent \emph{Упражнение} \theexnum\emph{.}~}



\theoremstyle{plain}
\newtheorem{theorem}{Теорема}
\newtheorem{lemma}{Лемма}
\newtheorem{proposition}{Предложение}
\newtheorem{corollary}{Следствие}
\theoremstyle{definition}
\newtheorem{definition}{Определение}
\theoremstyle{remark}
\newtheorem{remark}{Замечание}
\newtheorem{example}{Пример}

\textheight=250mm %
\textwidth=180mm %
\oddsidemargin=-10.4mm%
\evensidemargin=-10.4mm %
\topmargin=-24.4mm


\begin{document}
	\pagestyle{empty}
    \name{}
	\head{6 класс}{Мать ученья}
	\bigskip
	
\task Заметим, что если перевернуть лист, на котором написаны цифры, то цифры $0, 1, 8$ не изменятся, $6$ и $9$ поменяются местами, а остальные потеряют смысл. Сколько существует девятизначных чисел, которые при переворачивании листа не изменяются?\\

\task  Четыре семьи, в каждой из которых $4$ человека пришли в кинотеатр. Сколькими способами они могут усесться в ряду с $16$-ю креслами, так чтобы члены каждой семьи сидели подряд?\\

\task Сколькими способами можно расставить в ряд $20$ учеников группы, чтобы 
    \begin{itemize}
        \item[а)] Андрей и Борис стояли рядом?
        \item[б)] чтобы Борис стоял левее Андрея (не обязательно рядом)?
    \end{itemize}

\task Сколькими способами можно разбить $20$ человек на пары?\\

\task На прямой отмечено $12$ точек, а на параллельной ей прямой — $34$ точки. Сколько существует
    \begin{itemize}
        \item[а)] треугольников с вершинами в этих точках?
        \item[б)] четырехугольников с вершинами в этих точках?
    \end{itemize}

\task В столовой сварили $n$ одинаковых макаронин и их надо раздать $k$ голодным школьникам. Сколькими способами можно это сделать, если не обязательно все школьники должны получить хоть одну макаронину?\\

\task Сколькими способами из $50$ человек, среди которых $25$ юношей и $25$ девушек, можно выбрать компанию в которой одинаковое число юношей и девушек?\\

\end{document}
