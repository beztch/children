\documentclass[a4paper,12pt]{extreport}
\usepackage[utf8]{inputenc}
\usepackage[russian]{babel}
\usepackage{ragged2e}
\usepackage[mag=1000,a4paper,left=1cm,right=1cm,top=1cm,bottom=1cm,noheadfoot]{geometry}
\usepackage{mathtext}
\usepackage{amsmath,amssymb,amsthm,amscd,amsfonts,graphicx,epsfig,textcomp,wrapfig}
\usepackage[dvips]{graphicx}
\graphicspath{{noiseimages/}}

\newcommand{\tab}{\hspace{10mm}}
\newcommand{\name}{
			\normalsize
			{
				\ \ \quad
                \mbox{} \hfil {\flushleft{взлет МО}} \hfill {07.09.2023}
			%\quad
			}
			\vspace{5pt}\hrule
		}

\def\head#1#2{
	\begin{center}{
			\LARGE
			\bf #2
			
			{\normalsize \bf #1}
			\vspace{-10pt}
	}\end{center}
}

\newcommand{\del}{\mathop{\raisebox{-2pt}{\vdots}}}
\newcommand{\q}{}

\newcounter{tasknum}
\setcounter{tasknum}{0}
\def\thetasknum{{\textbf{\arabic{tasknum}}}}
\newcommand{\task}{\refstepcounter{tasknum}\vspace{2pt}\noindent \textbf{} \thetasknum\textbf{.}~}
\newcommand{\coff}{\refstepcounter{tasknum}\vspace{2pt}\noindent \textbf{Задача} \thetasknum *\textbf{.}~}


\newcommand{\eq}[1]{\underset{#1}{\equiv}}
\newcommand{\dv}{\ensuremath{\mathop{\raisebox{-2pt}{\vdots}}}}
\newcommand{\ndv}{\not \dv}

\newcounter{exnum}
\setcounter{exnum}{0}
\def\theexnum{{\emph{\arabic{exnum}}}}
\newcommand{\ex}{\refstepcounter{exnum}\vspace{2pt}\noindent \emph{Упражнение} \theexnum\emph{.}~}



\theoremstyle{plain}
\newtheorem{theorem}{Теорема}
\newtheorem{lemma}{Лемма}
\newtheorem{proposition}{Предложение}
\newtheorem{corollary}{Следствие}
\theoremstyle{definition}
\newtheorem{definition}{Определение}
\theoremstyle{remark}
\newtheorem{remark}{Замечание}
\newtheorem{example}{Пример}

\textheight=250mm %
\textwidth=180mm %
\oddsidemargin=-10.4mm%
\evensidemargin=-10.4mm %
\topmargin=-24.4mm


\begin{document}
	\pagestyle{empty}
    \name{}
	\head{8 класс}{Вокруг остатков}
	\bigskip
	
\task Найдите остаток от деления

\begin{enumerate}
    \item[а)] $7^{143}$ на $10$
    \item[б)] $5^{315}$ на $7$
    \item[в)] $7^{7^{7}}$ на $11$
\end{enumerate}
\\

\task $5x + 8y$ дает остаток $1$ при делении на $13$. Найдите остатки от деления на $13$ у
\begin{enumerate}
    \item[а)] $5x + 60y$
    \item[б)] $18x - 31y$
    \item[в)] $2x - 2y$
\end{enumerate}
\\

\task Дано $101$ натуральное число. Докажите, что из них можно выбрать несколько, сумма которых делится на $101$.\\

\task Найдите все тройки простых чисел такие, что все три их попарные разности тоже являются простыи числами.\\

\task Докажите, что из любых $60$ натуральных чисел можно выбрать $58$ таких, что $a_{1} + 2a_{2} + 3a_{3} + ... + 58a_{58}$ делится на 59.\\

\task Пусть $p$ -- простое, $a$ не делится на $p$. Докажите, что у чисел $a, 2a, 3a ... (p - 1)a$ попарно разные остатки при делении на $p$.\\

\task (Малая теорема Ферма) Докажите, что в условиях предыдущей задачи $a^{p - 1} \equiv 1 $ (mod p).\\

\task Пусть $p$ и $q$ -- простые числа. Докажите, что $p^{q} + q^{p} - p - q$ делится на $pq$.\\

\end{document}
