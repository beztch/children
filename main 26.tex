\documentclass[a4paper,12pt]{extreport}
\usepackage[utf8]{inputenc}
\usepackage[russian]{babel}
\usepackage{ragged2e}
\usepackage[mag=1000,a4paper,left=1cm,right=1cm,top=1cm,bottom=1cm,noheadfoot]{geometry}
\usepackage{mathtext}
\usepackage{amsmath,amssymb,amsthm,amscd,amsfonts,graphicx,epsfig,textcomp,wrapfig}
\usepackage[dvips]{graphicx}
\graphicspath{{noiseimages/}}

\newcommand{\tab}{\hspace{10mm}}
\newcommand{\name}{
			\normalsize
			{
				\ \ \quad
                \mbox{} \hfil {\flushleft{Взлет}} \hfill {26.01.2024}
			%\quad
			}
			\vspace{5pt}\hrule
		}

\def\head#1#2{
	\begin{center}{
			\LARGE
			\bf #2
			
			{\normalsize \bf #1}
			\vspace{-10pt}
	}\end{center}
}

\newcommand{\del}{\mathop{\raisebox{-2pt}{\vdots}}}
\newcommand{\q}{}

\newcounter{tasknum}
\setcounter{tasknum}{0}
\def\thetasknum{{\textbf{\arabic{tasknum}}}}
\newcommand{\task}{\refstepcounter{tasknum}\vspace{2pt}\noindent \textbf{} \thetasknum\textbf{.}~}
\newcommand{\coff}{\refstepcounter{tasknum}\vspace{2pt}\noindent \textbf{Задача} \thetasknum *\textbf{.}~}


\newcommand{\eq}[1]{\underset{#1}{\equiv}}
\newcommand{\dv}{\ensuremath{\mathop{\raisebox{-2pt}{\vdots}}}}
\newcommand{\ndv}{\not \dv}

\newcounter{exnum}
\setcounter{exnum}{0}
\def\theexnum{{\emph{\arabic{exnum}}}}
\newcommand{\ex}{\refstepcounter{exnum}\vspace{2pt}\noindent \emph{Упражнение} \theexnum\emph{.}~}



\theoremstyle{plain}
\newtheorem{theorem}{Теорема}
\newtheorem{lemma}{Лемма}
\newtheorem{proposition}{Предложение}
\newtheorem{corollary}{Следствие}
\theoremstyle{definition}
\newtheorem{definition}{Определение}
\theoremstyle{remark}
\newtheorem{remark}{Замечание}
\newtheorem{example}{Пример}

\textheight=250mm %
\textwidth=180mm %
\oddsidemargin=-10.4mm%
\evensidemargin=-10.4mm %
\topmargin=-24.4mm


\begin{document}
	\pagestyle{empty}
    \name{}
	\head{группа 9.1}{Тут Что, Конструктивы?}
	\bigskip

\task Докажите, что уравнение $x^2 + y^2 - z^2 = 2024$ имеет бесконечно много решений в целых числах.
\\

\task Существуют ли $100$ рациональных чисел таких, что произведение любых двух не является целым числом, а произведение любых трёх является целым?\\
	
\task Существует ли $10$ таких различных целых чисел, что сумма любых $9$ из них является полным квадратом?\\

\task Существует ли арифметическая прогрессия $a_1, a_2, \dots , a_{2024}$ с ненулевой разностью такая, что каждый её член имеет вид $\frac{1}{n}$?\\

\task Существуют ли три попарно различных ненулевых целых числа, сумма которых равна нулю, а сумма тринадцатых степеней которых является квадратом некоторого натурального числа?\\

\task Существуют ли такие натуральные числа $a, b$ и $c$, что $a^2 - 1$ делится на $b$, $b^2 - 1$ делится на $c$ и $c^2 - 1$ делится на $a$, причем $a + b + c > 2024$?\\

\task Существует ли такой набор из $1000$ различных натуральных чисел, что для любых двух чисел из набора их сумма делится на их разность?\\

\task Числа m и n назовём похожими друг на друга, если у них совпадают множества простых делителей. Числа m и n назовём очень похожими, если во-первых, m и n похожи друг на друга, а во-вторых, m + 1 и n + 1 похожи друг на друга. Докажите, что пар очень похожих натуральных чисел бесконечно много.\\

\task Существует ли такое натуральное $N$, что в любой возрастающей арифметической прогрессии, состоящей из $N$ натуральных чисел, найдется число, кратное некоторому простому больше $10000$?\\

\end{document}
