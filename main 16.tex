\documentclass[a4paper,12pt]{extreport}
\usepackage[utf8]{inputenc}
\usepackage[russian]{babel}
\usepackage{ragged2e}
\usepackage[mag=1000,a4paper,left=1cm,right=1cm,top=1cm,bottom=1cm,noheadfoot]{geometry}
\usepackage{mathtext}
\usepackage{amsmath,amssymb,amsthm,amscd,amsfonts,graphicx,epsfig,textcomp,wrapfig}
\usepackage[dvips]{graphicx}
\graphicspath{{noiseimages/}}

\newcommand{\tab}{\hspace{10mm}}
\newcommand{\name}{
			\normalsize
			{
				\ \ \quad
                \mbox{} \hfil {\flushleft{взлет МО}} \hfill {10.10.2023}
			%\quad
			}
			\vspace{5pt}\hrule
		}

\def\head#1#2{
	\begin{center}{
			\LARGE
			\bf #2
			
			{\normalsize \bf #1}
			\vspace{-10pt}
	}\end{center}
}

\newcommand{\del}{\mathop{\raisebox{-2pt}{\vdots}}}
\newcommand{\q}{}

\newcounter{tasknum}
\setcounter{tasknum}{0}
\def\thetasknum{{\textbf{\arabic{tasknum}}}}
\newcommand{\task}{\refstepcounter{tasknum}\vspace{2pt}\noindent \textbf{} \thetasknum\textbf{.}~}
\newcommand{\coff}{\refstepcounter{tasknum}\vspace{2pt}\noindent \textbf{Задача} \thetasknum *\textbf{.}~}


\newcommand{\eq}[1]{\underset{#1}{\equiv}}
\newcommand{\dv}{\ensuremath{\mathop{\raisebox{-2pt}{\vdots}}}}
\newcommand{\ndv}{\not \dv}

\newcounter{exnum}
\setcounter{exnum}{0}
\def\theexnum{{\emph{\arabic{exnum}}}}
\newcommand{\ex}{\refstepcounter{exnum}\vspace{2pt}\noindent \emph{Упражнение} \theexnum\emph{.}~}



\theoremstyle{plain}
\newtheorem{theorem}{Теорема}
\newtheorem{lemma}{Лемма}
\newtheorem{proposition}{Предложение}
\newtheorem{corollary}{Следствие}
\theoremstyle{definition}
\newtheorem{definition}{Определение}
\theoremstyle{remark}
\newtheorem{remark}{Замечание}
\newtheorem{example}{Пример}

\textheight=250mm %
\textwidth=180mm %
\oddsidemargin=-10.4mm%
\evensidemargin=-10.4mm %
\topmargin=-24.4mm


\begin{document}
	\pagestyle{empty}
    \name{}
	\head{7а класс}{матИгры}
	\bigskip
	
\task Имеются две кучки камней по $200$ штук в каждой. За ход разрешается брать любое количество камней, но только из одной кучки. Проигрывает тот, кому нечего брать. Кто выиграет при правильной игре?
\\

\task Имеются две кучки камней, в одной $200$ штук, во второй - $300$. За ход разрешается брать любое количество камней, но только из одной кучки. Проигрывает тот, кому нечего брать. Кто выиграет при правильной игре?
\\

\task Камушки лежат в одиннадцати кучах. В каждой куче по $30$ камней. За один ход можно взять любое число камней из любой одной кучи. Кто выигрывает при правильной игре?
\\

\task На концах клетчатой полоски $1 \times 20$ стоит по шашке. Играют двое; за ход разрешается сдвинуть одну шашку в направлении другой на одну или на две клетки. Перепрыгивать шашкой через шашку нельзя. Проигрывает тот, кто не может сделать ход. Кто выиграет при правильной игре?
\\

\task Есть а) $100$  б) $102$ сырника. Два игрока съедают по-очереди от $1$ до $4$ сырников. Выиграет тот, кто съест последний сырник. Кто выиграет при правильной игре?
\\

\task Петя и Вася играют в игру. У них есть $2020$ бананов. Сначала Вася распределяет бананы по трем кучам, в каждую хотя бы по одному банану. Затем они делают ходы по очереди, начинает Петя. За один ход можно взять из одной кучи $1$, $2$ или $3$ банана. Проигрывает не имеющий хода. Кто выигрывает при правильной игре?\\

\task  а) В строчку написано несколько минусов. Два игрока по очереди переправляют один или два соседних минуса на плюс; выигрывает переправивший последний минус.

б) Рассмотрите вариант предыдущей игры, в котором минусы написаны не в строчку, а стоят по кругу (и исправлять можно один или два стоящих рядом минуса)\\

\task Двое играют на доске а)$2000 \times 22$ б)$2000 \times 23$ клеток. Каждый по очереди отмечает квадрат по линиям сетки (любого возможного размера) и закрашивает его. Выигрывает тот, кто закрасит последнюю клетку. Дважды закрашивать клетки нельзя. Кто выиграет при правильной игре и как надо играть?\\

\end{document}
